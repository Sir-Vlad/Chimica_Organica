\section{Acidi e Basi di Br{\o}nsted-Lowry}
La teoria più utilizzata in ambito chimico organico è la teoria di Br{\o}nsted-Lowry, che dice:
\begin{itemize}
	\item[] \textbf{Acido} \(\rightarrow\) sostanza che dona un protone
	\item[] \textbf{Base} \(\rightarrow\) sostanza che accetta un protone
\end{itemize}

Questa teoria implica la presenta contemporanea di un acido e una base in soluzione. Si determinano così delle coppie, formate ciascuna da un acido e da una base, chiamate \textbf{coppie coniugate acido-base}. Si può affermare che: un acido, perdendo un protone, si trasforma nella sua \textbf{base coniugata}, e analogamente, una base che acquista un protone si trasforma nel proprio \textbf{acido coniugato}. Si possono trovare alcune coppie coniugate nell'\hyperref[ap:acidi.basi]{appendice II}.

\begin{figure}[H]
	\centering
	\vspace{7mm}
	\ch{
	2 "\OX{o1,Na}" + "\OX{r1,Cl}" {}2
	->
	2 "\OX{o2,Na}" {}+ + 2 "\OX{r2,Cl}" {}-
	}
	\redox(o1,o2)[draw=red,->]{\small \color{red}coppia coniugata}
	\redox(r1,r2)[draw=blue,->][-1]{\small \color{blue}coppia coniugata}
	\vspace{7mm}
\end{figure}

\noindent Gli acidi e le basi di Br{\o}nsted-Lowry hanno delle caratteristiche:
\begin{itemize}
	\item Un acido può avere carica positiva, neutra o negativa
	\item Una base può avere carica positiva, neutra o negativa
	\item Gli acidi sono classificati come monoprotici, biprotici o triprotici, in base al numero di protoni possono cedere
	\item Le molecole che possono comportarsi sia da base che da acido vengono chiamati \textbf{anfoteri}
	\item Esiste una relazione tra la forza dell'acido e la forza della sua base coniugata, ovvero più è forte l'acido, più debole sarà la sua base coniugata
\end{itemize}


