\section{Acidi e Basi di Arrhenius}
\noindent Secondo la definizione di \textbf{Arrhenius}:
\begin{itemize}
	\item[] \textbf{Acido} \(\rightarrow\) sostanza che si scioglie in acqua producendo ioni \ch{H+}
	\item[] \textbf{Base} \(\rightarrow\) sostanza che si scioglie in acqua producendo ioni \ch{OH-}
\end{itemize}
Questa definizione è valida solo se lavoriamo in ambiente acquoso.

Quando un acido si scioglie in acqua, esso reagisce con l'acqua per produrre \ch{H+}/\ch{H3O+}. Per esempio, quando \ch{HCl} si scioglie in acqua reagisce con l'acqua per dare ione ossonio e ione cloruro:


\begin{center}
	\ch[circled=all,circletype=math]{
	H2O_{(l)} + HCl_{(l)} -> !(ossonio)( H3O^{\color{red}+}_{(aq)} ) + Cl^{\color{blue}-}_{(aq)}
	}
\end{center}

Per mostrare il trasferimento di un protone dall'acido all'acqua si utilizza una \textbf{freccia curva}. Innanzitutto, si scrivono i reagenti e i prodotti come strutture di Lewis. Poi si utilizzano le frecce per far vedere lo spostamento degli elettroni.

\begin{reactions}
	\AddRxnDesc{Autoprotolisi dell'acqua}
	\chemfig{@{O}\charge{45=\:,135=\:}{O}(-[6]H)(-[4]H)} 
	\+ 
	\chemfig{@{H}H-[@{ClL}]@{Cl}\charge{0:1pt=\:,90:1pt=\:,270:1pt=\:}{Cl}} 
	\arrow(.mid east--.mid west)
	\chemfig{\charge{90:1pt=\:,90:7pt=\chargeColor{+}}{O}(-[6]H)(-[4]H)(-[0]H)} 
	\+ 
	\charge{0:1pt=\:,90:1pt=\:,180:1pt=\:,270:1pt=\:,45:3pt=\chargeColor{-}}{Cl}
	\chemmove[green!60!black!70]{
		\draw[shorten <=3pt, shorten >= 1pt] (O).. controls +(45:1cm) and +(90:.6cm) .. (H);
		\draw[shorten <=3pt, shorten >= 1pt] (ClL).. controls +(-90:.5cm) and +(240:.5cm) .. (Cl);
	}
\end{reactions}

Il concetto di acidi e basi di Arrhenius è legato a reazioni in ambiente acquoso e non è per niente adatto a trattare reazioni non in ambiente acquoso. Per questo motivo per le reazioni acido-base di composti organici utilizzeremo la definizione di acido e basi di Br{\o}nsted-Lowry.