\section{Ibridazione degli orbitali}
Un legame covalente, tra due atomi, si forma sovrapponendo una porzione di un orbitale di un atomo con una porzione di orbitale dell'altro atomo.

Se la sovrapposizione degli orbitali è frontale si parla di \textbf{legame sigma} (\(\upsigma\)) mentre se è laterale si parla di \textbf{legame pi} (\(\uppi\)).

La formazione di legami covalenti di carbonio, azoto e ossigeno presentano il problema che i legami dovrebbero formare angoli di \ang{90} ma questo non accade, per spiegare questo fenomeno si utilizzano gli \textbf{orbitali ibridi}.

\begin{figure}[H]
	\centering
	\begin{modiagram}[style=round]
		\AO{s}[label={\(2s\)}]{0}
		\AO{p}[label[x]={\(2p_x\)},label[y]={\(2p_y\)},label[z]={\(2p_z\)}]{1.5;up,up}
	\end{modiagram}\quad
	\tikz{\draw[->](0,1)--(2,1);\draw[white](0,0)--(0,1)}\quad
	\begin{modiagram}[style=round]
		\AO{s}[label={\(2s\)}]{0;up}
		\AO{p}[label[x]={\(2p_x\)},label[y]={\(2p_y\)},label[z]={\(2p_z\)}]{1.5;up,up,up}
	\end{modiagram}
\end{figure}

\begingroup
\setchemfig{
	atom sep = 3em,
	double bond sep=3pt
}
\subsection{Ibridazione \texorpdfstring{\(sp^3\)}{sp3}}
La combinazione di un orbitale \(2s\) e tre orbitali \(2p\) porta alla formazione di \textbf{quattro orbitali ibridi} \(\mathbf{sp^3}\) equivalenti. I quattro orbitali ibridi si dispongono ai vertici di un tetraedro con un angolo di legame di \ang{109.5}.

\begin{center}
	\chemfig{@{2}C(-[@{h1}2]H)(-[@{h2}5]H)(<[@{h3}7]H)(<:[8]H)}
	\namebond{h1}{2}{\scriptsize\color{red}1,1\angstrom}{-7}
	\arclabel{0.75cm}{h2}{2}{h3}{\footnotesize\color{red}109.5\textdegree}{0}{0}
\end{center}

\subsection{Ibridazione \texorpdfstring{\(sp^2\)}{sp2}}
La combinazione di un orbitale \(2s\) e due orbitali \(2p\) porta alla formazione di \textbf{tre orbitali ibridi} \(\mathbf{sp^2}\) equivalenti. I tre orbitali ibridi si dispongono sul piano diretti verso i vertici di triangolo equilatero con un angolo di legame di \ang{120}.

\begin{center}
	\chemfig{@{2}C(-[@{h1}3]H)(-[@{h2}5]H)=@{1}C(-[1]H)(-[7]H)}
	\namebond[-7pt]{2}{1}{\scriptsize\color{red}1,33\angstrom}{0}
	\namebond[-7pt]{h2}{2}{\scriptsize\color{red}1,07\angstrom}{-5}

	\arclabel{0.75cm}{h1}{2}{1}{\footnotesize\rotatebox{202}{\color{red}121.7\textdegree}}{-16}{-16}
	\arclabel{0.75cm}{h1}{2}{h2}{\footnotesize\rotatebox{90}{\color{red}116.6\textdegree}}{-12}{0}
\end{center}

\subsection{Ibridazione \texorpdfstring{\(sp\)}{sp}}
La combinazione di un orbitale \(2s\) e un orbitali \(2p\) porta alla formazione di \textbf{tre orbitali ibridi} \(\mathbf{sp}\) equivalenti. I due orbitali sono orientati in modo da formare un angolo di legame di \ang{180}.

\begin{center}
	\chemfig{@{1}C(-[@{h1}4]H)~@{2}C(-[@{h2}0]H)}
	\namebond[-8pt]{2}{1}{\scriptsize\color{red}1,2\angstrom}{0}
	\namebond{h2}{2}{\scriptsize\color{red}1,06\angstrom}{6}
	\arclabel{0.5cm}{h1}{1}{2}{\footnotesize\rotatebox{180}{\color{red}180\textdegree}}{-10}{0}
\end{center}

\endgroup


