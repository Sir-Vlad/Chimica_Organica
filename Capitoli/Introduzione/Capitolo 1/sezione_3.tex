\section{Geometria molecolare}
Molte caratteristiche di molecole organiche come punto di ebollizione e di fusione, solubilità, odore e sapore sono riconducibili alla loro geometria molecolare.

\subsection{Modello VSEPR}
Il modello \textbf{\Ac{VSEPR}} permette di determinare la struttura di una molecola considerando esclusivamente le forze repulsione che agiscono tra le coppie elettroniche del livello di valenza dell'atomo che occupa il centro della molecola stessa.

\subsubsection{Trovare le strutture molecolari con il metodo VSEPR}
\noindent Per determinare la struttura di un composto è opportuno procedere come segue:
\begin{enumerate}
	\item scrivere la struttura di Lewis degli atomi del composto considerato
	\item contare quante coppie di elettroni sono presenti attorno all'atomo centrale, distinguendo le coppie di legame da quelle libere e contare i legami multipli come unico centro di repulsione
	\item ricercare nella~\autoref{tab:strutture_molecolari} quale disposizione elettronica porta alla massima distanza le coppie che sono individuate
\end{enumerate}

\begingroup
\begin{table}[H]
	\centering
	\newcommand{\minitab}[2][l]{\begin{tabular}{#1}#2\end{tabular}}
	\newcommand{\molecola}[1]{\renewcommand{\arraystretch}{.5}\begin{tabular}{c}#1\end{tabular}}
	\rowcolors{2}{gray!15}{}
	\begin{tabular}{lcccccc}
		\toprule
		\multirow{2}{*}{\minitab[c]{Distribuzione                            \\ elettroni}} & \multirow{2}{*}{\minitab[c]{Struttura\\ molecola}}  & \multirow{2}{*}{\minitab[c]{Formula\\ generale}} & \multicolumn{3}{c}{Coppie elettroniche} & \multirow{2}{*}{\minitab[c]{Angoli}} \\
		            &                   &   & totali & di legame & solitarie \\
		\midrule
		Lineare     & \molecola{                                             \\ \chemfig{B-A-B}\\ \\} & \chemfig{AB_2} & 2      & 2         & 0    & \ang{180}     \\
		Planare     & \molecola{                                             \\ \chemfig{B-[:30]A(-[2]B)-[:-30]B}\\ \\} & \chemfig{AB_3}  & 3      & 3         & 0    & \ang{120}     \\
		Angolata    & \molecola{                                             \\ \chemfig{B-[:30]\charge{90:2pt=\:}{A}(-[2,0.4,,,draw=none])-[:-30]B}\\ \\} & \chemfig{AB_2E} & 3      & 2         & 1   & $<$ \ang{120}      \\
		Tetraedrica & \molecola{                                             \\ \chemfig{B>:[:30]A(-[2]B)(<[:250]B)-[:-30]B}\\ \\}
		            & \chemfig{AB_4}    & 4 & 4      & 0       & \ang{109.5}              \\
		Piramidale  & \molecola{                                             \\ \chemfig{B>:[:30]\charge{90:2pt=\:}{A}(-[2,0.4,,,draw=none])(<[:250]B)-[:-30]B}\\ \\}                            & \chemfig{AB_3E} & 4      & 3         & 1    & $<$ \ang{109.5}     \\
		Angolata    & \molecola{                                             \\ \chemfig{B>:[:30]\charge{90:2pt=\:,0:2pt=\:}{A}(-[2,0.4,,,draw=none])(<[:250]B)-[:-30,,,,draw=none]}\\ \\}

		            & \chemfig{AB_2E_2} & 4 & 2      & 2       & $\ll$ \ang{109.5}              \\
		\bottomrule
	\end{tabular}
	\caption{Strutture molecolari prevedibili con il metodo VSEPR}\label{tab:strutture_molecolari}
\end{table}
\endgroup

\subsection{Risonanza}
Esistono molecole la cui struttura e il cui comportamento non sono correttamente descritti dalla formula di Lewis.

Ad esempio, si è notato sperimentalmente che lo ione carbonato \ch{CO3-} ha tutti e tre i legami della stessa lunghezza, in particolare, ogni legame è una via di mezzo tra un legame singolo e un legame doppio come lunghezza. Per spiegare questo fenomeno si ricorre alla teoria della risonanza.

\subsubsection{Teoria della risonanza} 
Secondo questa teoria, molte molecole e ioni sono meglio descritti mediante due o più strutture di Lewis, considerando la reale molecole o ione l'unione di tutte le strutture. Le singole strutture sono chiamate \textbf{strutture limite di risonanza}. Per mostrare che la reale molecola o ione è \textbf{ibrido di risonanza} delle varie strutture limite di risonanza, queste vengono interconnesse tramite \textbf{frecce a doppia punta} (\(\longleftrightarrow\)).

\begin{figure}[H]
	\centering
	\begingroup
	\schemestart
	\chemleft{[}
	\subscheme{
	\chemfig{
	CH_3-C(-[7]\charge{45=\:,225=\:,315=\:,0:5pt=\chargeColor{-}}{O})(=[@{b1}1]@{o1}\charge{45=\:,135=\:}{O})}
	\arrow{<->}
	\chemfig{CH_3-\charge{0:3pt=\chargeColor{+}}{C}(-[@{o2}7]@{b2}\charge{45=\:,225=\:,315=\:,0:5pt=\chargeColor{-}}{O})(-[1]\charge{45=\:,135=\:,315=\:,0:5pt=\chargeColor{-}}{O})}
	\arrow{<->}
	\chemfig{CH_3-C(-[1]\charge{45=\:,135=\:,315=\:,0:5pt=\chargeColor{-}}{O})(=[7]\charge{225=\:,315=\:}{O})}
	\chemmove[green!60!black!70]{
		\draw[shorten <=3pt,shorten >=1pt](b1)..controls +(-45:.7cm) and +(-45:.7cm) ..(o1);
		\draw[shorten <=3pt,shorten >=1pt](b2)..controls +(45:.5cm) and +(35:.5cm)..(o2);
	}
	}
	\chemright{]}
	\schemestop
	\endgroup
	\caption{Forme di risonanza dell'acido acetico (\ch{CH3COOH})}
\end{figure}

\subsubsection{Regole per scrivere le forme di risonanza}
\noindent Per scrivere strutture limite di risonanza accettabili, bisogna osservare le seguenti regole:
\begin{enumerate}
	\item Tutte le strutture limite devono avere lo stesso numero di elettroni di valenza
	\item Tutte le strutture limite devono rispettare le regole del legame covalente, quindi nessuna struttura limite deve avere
	      \begin{itemize}
		      \item più di 2 elettroni nel guscio di valenza dell'idrogeno
		      \item più di 8 elettroni per gli elementi del secondo periodo
		      \item più di 12 elettroni per gli elementi del terzo periodo
	      \end{itemize}
	\item Le posizioni di tutti i nuclei devono restare invariate; le strutture limite devono variare solo per disposizione degli elettroni
	\item Tutte le strutture limite devono avere lo stesso numero totale di elettroni accoppiati e non
\end{enumerate}