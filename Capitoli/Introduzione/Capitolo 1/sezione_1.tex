\section{Struttura dell'atomo}
Gli atomi sono costituiti da un nucleo circondato da \textbf{elettroni}, particelle cariche negativamente. Il nucleo, a sua volta, è formato da \textbf{protoni}, particelle cariche positivamente, e \textbf{neutroni}, particelle neutre.

Il \textit{numero atomico} di un elemento è il numero di protoni presenti nel nucleo, mentre \textit{peso atomico} è la somma delle masse dei protoni e neutroni contenuti nel nucleo.

\subsection{Gli orbitali Atomici}
Secondo il modello probabilistico, gli elettroni vengono considerati come delle onde. La zona dove c'è più probabilità di trovare queste onde viene chiamato \textbf{orbitale atomico}.

Gli orbitali hanno delle caratteristiche descritte dai numeri quantici, i quali sono:
\paragraph{Numero quantico principale (\(n\))} Indica l'energia e la dimensione dell'orbitale. Può assumere valori da 1 a massimo 7 e il numero totale di orbitali presenti nel livello \(n\) è \(n^2\). Gli orbitali che condividono lo stesso valore di \(n\) vengono raggruppati in \textbf{gusci}.
\paragraph{Numero quantico secondario (\(\ell\))} Indica la forma dell'orbitale. Può assumere valori da 0 a \(n-1\). Gli orbitali che condividono lo stesso valore di \(\ell\) è chiamato \textbf{sottolivello}.
\paragraph{Numero quantico magnetico (\(m\))} Indica orientamento dell'orbitale. Può assumere valori da \(-\ell\) a \(+\ell\).
\paragraph{Numero quantico di spin (\(s\))} Indica il senso di rotazione dell'elettrone sul proprio asse (spin). Può assumere i valori \(+\nicefrac{1}{2}\) e \(-\nicefrac{1}{2}\).

\subsection{Configurazione Elettronica degli atomi}
La \textbf{configurazione elettronica} di un atomo è una descrizione di come si sono disposti gli elettroni negli orbitali. É possibile determinare la configurazione elettronica di un atomo usando le seguenti regole:
\begin{description}
	\item[Regola 1:] \textit{Gli orbitali si riempiono in ordine di energia crescente, da quello ad energia più bassa a quello ad energia più alta.}
	\item[Regola 2:] \textit{Ciascun orbitale può contenere fino a due elettroni con i loro spin opposti.}
	\item[Regola 3:] \textit{Quando sono disponibili orbitali di uguale energia ma non ci sono elettroni sufficienti a riempirli completamente, allora un solo elettrone viene aggiunto a ciascun degli orbitali equivalenti prima di aggiungere un secondo elettrone a uno qualsiasi di essi.}
\end{description}
