\section{Definizione e classificazione}
I carboidrati sono biomolecole che hanno formula generale \ch{C_n(H2O)_n}. I carboidrati sono poliidrossialdeidi o poliidrossichetoni o sostanze che per idrolisi danno composti di questo tipo.

I carboidrati vengono classificati a seconda della loro struttura, come:
\begin{itemize}
	\item \hyperref[sec:monosaccardi]{monosaccaride}
	\item \hyperref[sec:disaccaridi]{disaccaride} (due unità di monosaccaride)
	\item \hyperref[sec:polisaccaridi]{polisaccaridi} (più di due unità di monosaccaride)
\end{itemize}
Le tre classi di carboidrati sono in relazione tramite la reazione di idrolisi:
\begin{reaction}
	\AddRxnDesc{Interconversione tra le classi dei carboidrati}
	Polisaccaridi \arrow{->[\ch{H2O}][\Hpiu{1}]}
	Disaccaridi \arrow{->[\ch{H2O}][\Hpiu{1}]}
	Monosaccaride
\end{reaction}

%%%%%%%%%%%%%%%%%%%%%%%%%%%%%%%%%%%%%%%%%%%%%%%%%%%%%%%%%%%%%%%%%%%