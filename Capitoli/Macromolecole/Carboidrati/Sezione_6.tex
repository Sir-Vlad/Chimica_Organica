\section{Carboidrati derivati}


\subsection{Deossizuccheri}
I deossizuccheri sono zuccheri dove uno o più gruppi ossidrilici vengono sostituiti da atomi idrogeno. Il deossizucchero più importante è il \iupac{2-deossiribosio}, il quale è uno dei componenti del DNA. In questo caso l'ossidrile che manca è quello del carbonio in posizione 2.

\begin{figure}[H]
	\centering
	\chemname{\chemfig{O?-[:-30,1.8](-[2]OH)<[:240,1.5]-[4,1.5,,,line width=4pt,line cap=round](-[6]OH)>[:120,1.5]?(-[2,,,4]HOH_2C)}}{\iupac{\b-\D-deossiribosio}}
\end{figure}

\subsection{Fosfati degli zuccheri}
Gli esteri fosfati dei monosaccaridi sono presenti in tutte le cellule viventi come intermedi nel metabolismo dei carboidrati. I fosfati del ribosio e del \iupac{2-deossiribosio} formano insieme alle basi azotate, la struttura del DNA e del RNA.

\begin{figure}[H]
	\centering
	\chemname{\chemfig{!{ATP}}}{\iupac{adonosina 5'-trifosfato} (ATP)}
\end{figure}

\subsection{Amminozuccheri}
Negli amminozuccheri un ossidrile dello zucchero è sostituito da un gruppo amminico; nella maggior parte dei casi il gruppo amminico è acetilato. La \iupac{\D-glucosammina} è uno degli amminozuccheri più diffusi.

\begin{figure}[H]
	\centering
	\chemname{\chemfig{O?-[7,1.5](-[0,,,,decorate,decoration={snake}]OH)<[5,1.5](-[6]NH_2)-[4,1.5,,,line width=4pt,line cap=round](-[2]OH)>[3,1.5](-[6]OH)-[1,1.5]?(-[2]CH_2OH)}}{\iupac{\D-glucosammina}}
\end{figure}