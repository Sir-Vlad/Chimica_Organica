\section{Fenoli}
I fenoli a gli alcoli avendo lo stesso gruppo funzionale hanno molte proprietà in comune. Tuttavia, nel caso dei fenoli è molto difficile staccare il gruppo ossidrico dal resto della catena, al contrario degli alcoli.

L'ossidrile del fenolo può essere protonato, ma la successiva perdita di una molecola d'acqua porterebbe la catione fenile, che per sua natura è molto instabile e porterebbe la reazione a riformare il fenolo.

Di conseguenza, i fenoli non possono sostituire il gruppo ossidrilico con meccanismo \mech[1] e nemmeno con \mech[2], non potendo avere inversione di configurazione.

\subsection{Ossidazione dei fenoli}
I fenoli sono abbastanza facili da ossidare, nonostante l'assenza di un atomo d'idrogeno su carbonio che reca l'ossidrile.

Una delle categorie più importanti a livello biologico dei fenoli sono gli idrochinoni,fenoli con due gruppi ossidrilici. Questa categoria è importante per la loro interconversione con i chinoni nelle reazioni di ossidoriduzione.

\begin{reaction}
	\AddRxnDesc{Conversione tra idrochinone e chinone}
	\chemfig{*6(-(-[6]OH)=-=(-[2]OH)-=)}
	\arrow{->[\ch{Na2Cr2O7}][\ch{H2SO4}, \qty{30}{\celsius}]}[,1.8]
	\chemfig{*6(-(=[6]O)-=-(=[2]O)-=)}
\end{reaction}