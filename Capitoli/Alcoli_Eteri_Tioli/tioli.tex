\chapter{Tioli}
I tioli sono composti di formula chimica \ch{R-SH}. Il gruppo funzionale dei tioli è il \textbf{gruppo solfidrilico}, \ch{-SH}. I tioli si chiamano anche \textbf{mercaptani}.

Vengono preparati a partire dagli alogenuri alchilici, per sostituzione nucleofila con lo ione solfidrile.

\begin{reaction}
	\AddRxnDesc{Reazione di formazione dei tioli}
	\chemfig{R-X} \+ \ch[circled=all,circletype=math]{^{\color{blue}-}SH} \arrow \chemfig{R-SH} \+ \ch[circled=all,circletype=math]{X^{\color{blue}-}}
\end{reaction}

La caratteristica principale dei tioli è l'odore intenso e sgradevole. L'acidità dei tioli è maggiore di quella degli alcoli, per questo motivo possono essere tiolati facilmente per trattamento con una base acquosa.

\begin{reaction}
	\AddRxnDesc{Reazione di formazione dei tiolati}
	\ch[circled=all,circletype=math]{R-SH + Na^{\color{red}+}OH^{\color{blue}-} -> RS^{\color{blue}-}Na^{\color{red}+} + H2O}
\end{reaction}

\paragraph{Formazione di disolfuri}\label{par:disolfuri}
I tioli possono essere ossidati da agenti ossidanti blandi, come l'acqua ossigenata o lo iodio, in disolfuri, composti che contengono il legame \ch{S-S}.

\begin{reaction}
	\AddRxnDesc{Reazione di formazione dei disolfuri}
	\ch[circled=all,circletype=math]{2 R-SH <=>[ossidazione][riduzione] RS-SR}
\end{reaction}

Questa reazione è molto importante a livello biologico perché le proteine contengono ponti disolfuro che possono rotti per modificare la struttura proteica.