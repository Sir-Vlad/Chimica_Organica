\chapter{Eteri ed epossidi}
\section{Struttura}
Gli eteri sono composti con formula generale \ch{R-O-R'}, dove \ch{R} e \ch{R'} possono essere lo stesso gruppo alchilico o diverso. Gli epossidi, invece, sono eteri ciclici con un anello a tre termini.

\section{Proprietà fisiche}
Gli eteri sono molecole polari con una parziale carica negativa sull'ossigeno e una parziale carica positiva sui carboni legati all'ossigeno. A causa dell'ingombro sterico, le interazioni tra molecole sono molto deboli e per questo motivo, i punti di ebollizione degli eteri sono prossimi a quelli degli idrocarburi di peso molecolare comparabile.

Avendo una carica negativa sull'ossigeno, gli eteri formano legami a idrogeno con acqua e pertanto sono solubili in essa.

La caratteristica principale degli eteri è la loro inerzia chimica, ovvero non reagiscono con acidi, basi, agenti riducenti e ossidanti e sodio metallico. Grazie a questa proprietà vengono utilizzati come solventi nelle reazioni organiche.

Vengono utilizzati anche per estrarre composti organici dalle loro fonti naturali perché avendo una bassa temperatura di ebollizione possono essere eliminati per semplice distillazione.

\section{Preparazione degli eteri}
La preparazione degli eteri si differenzia a seconda se volgiamo ottenere un etere simmetrico o asimmetrico. Per preparare un \textbf{etere simmetrico} si fanno reagire due alcoli uguali in ambiente acido a caldo.
\begin{reaction}
	\AddRxnDesc{Preparazione eteri simmetrici}
	\ch{2 R-OH ->[ H2SO4 ][\(\Delta\)] R-O-R + H2O}
\end{reaction}

\noindent Mentre, per preparare un etere asimmetrico ci sono due metodi:
\begin{enumerate}
	\item reazione tra un alchene terziario con un alcol con catalizzatore
	      \begin{reaction}
		      \AddRxnDesc{Preparazione eteri asimmetrici}
		      \chemfig{C(-[3])(-[5])(=[0]C(-[1]R)(-[7]R))} \+ \ch{R-OH} \arrow{->[\ch[circled=all,circletype=math]{H^{\color{red}+}}]} \chemfig{C(-[2])(-[4]H)(-[6])(-[0]C(-[0]O-R)(-[2]R)(-[6]R))}
	      \end{reaction}
	\item \textbf{Sintesi di Williamson}, nella quale viene prima creato l'alcossido tramite trattamento con sodio metallico e poi il prodotto lo si fa reagire con un alogenuro alchilico tramite meccanismo \mech[2]
	      \begin{reactions}\label{rcn:sintesiWilliamson}
		      \AddRxnDesc{Sintesi di Williamson}
		      2 ROH \+ 2 Na &\arrow 2 \chemfig{\charge{35:3pt=\chargeColor{-}}{RO}-[,.8,,,white]\charge{35:3pt=\chargeColor{+}}{Na}} \+ H\(_2\)\\ \notag
		      \chemfig{\charge{35:3pt=\chargeColor{-}}{RO}-[,.8,,,white]\charge{35:3pt=\chargeColor{+}}{Na}} \+ \ch{R'-X} &\arrow ROR' + \chemfig{\charge{35:3pt=\chargeColor{-}}{Na}-[,.8,,,white]\charge{35:4pt=\chargeColor{+}}{X}}
	      \end{reactions}
	      Il processo funziona bene se \ch{R'} è un alogenuro alchilico primario visto che il secondo stadio è un processo \mech[2].
\end{enumerate}

\section{Scissione degli eteri}
Gli eteri sono basi di Lewis, grazie al suo doppietto non condiviso presente sull'ossigeno. Essi reagiscono sia con acidi protonici forti e sia con gli acidi di Lewis.

Nel caso in cui \ch{R} e/o \ch{R'} sono gruppi alchilici primari o secondari, la rottura del legame avviene ad opera di nucleofili forti come \ch{I-} o \ch{Br-} con meccanismo \mech[2].

\begin{reaction}
	\AddRxnDesc{Scissione degli eteri con acidi alogenidrici}
	\ch{R-O-R + HX -> R-X + R-OH}
\end{reaction}

\begin{reaction}
	\AddRxnDesc{Scissione degli eteri con tribromuro di boro}
	\ch{R-O-R} \arrow{->[1.\ \ch{BBr3}][2.\ \ch{H2O}]}[,1.5] \ch{RBr + ROH}
\end{reaction}

Se invece \ch{R} e/o \ch{R'} è terziario, non è più necessaria la presenza di un nucleofilo forte, perché la reazione decorre con il meccanismo \mech[1] (o \mech[e1]).

\section{Epossidi}
Gli epossidi (o ossirani) sono eteri ciclici con un anello a tre termini contenente un atomo di ossigeno. Sebbene gli epossidi siano classificati come esteri, la loro reattività è molto più elevata degli eteri.

\begin{figure}[H]
	\centering
	\setlength{\tabcolsep}{1cm}
	\renewcommand{\arraystretch}{2}
	\begin{tabular}{ccc}
		\chemfig{C?(-[3])(-[5])-[7]O-[1]C?(-[1])(-[7])}
		                  &
		\chemfig{H_2C?-[7]O-[1]C?H_2}
		                  &
		\chemfig{H_3CHC?-[7]O-[1]C?H_2}                             \\
		Epossido generale & Ossido di etilene & Ossido di propilene \\
	\end{tabular}
\end{figure}

\section{Sintesi degli epossidi dagli alcheni}
A livello industriale, l'epossido più importante è l'ossido di etilene e viene prodotto per ossidazione dell'etilene da parte dell'aria, usando l'argento come catalizzatore.

\begin{reaction}
	\AddRxnDesc{Sintesi degli epossidi dagli alcheni}
	\chemfig{CH_2=CH_2} \+ \ch{O2} \arrow(--.mid west){->[Ag][\qty{250}{\celsius}]}[,1.3]  \chemfig{H_2C?-[7]O-[1]C?H_2}
\end{reaction}

\section{Reazioni degli epossidi}
A causa della tensione angolare dell'anello a tre termini, gli epossidi sono molto reattivi degli eteri a catena lineare e danno prodotti di apertura.

Si può addizionare acqua all'anello, usando catalizzatore acido, per formare glicoli.
\begin{reaction}
	\AddRxnDesc{Addizione di acqua agli epossidi}
	\chemfig{RC?-[7]O-[1]C?R'} + \ch{H2O}
	\arrow(.mid east--.mid west){->[\ch[circled=all,circletype=math]{H^{\color{red}+}}]}
	\chemfig{RC(-[6,,2]OH)(-[0]CR'(-[6]OH))}
\end{reaction}

Si possono addizionare anche altri nucleofili agli epossidi in modo simile all'addizione di acqua.
\begin{reaction}
	\AddRxnDesc{Addizione di acqua agli epossidi}
	\chemfig{RC?-[7]O-[1]C?R'}
	\arrow(@c1.30--.180){0}[15,2.5]
	\chemfig{RC(-[6,,2]OH)(-[0]CR'(-[6]OR''))}
	\arrow(@c1.-30--.180){0}[-15,2.5]
	\subscheme{
	\chemfig{RC(-[6,,2]OH)(-[0]CR'-O-R''(-[6]OH))}
	}
	\chemmove[shorten <=3pt, shorten >= 3pt]{
	\tikzstyle{freccia}=[->,>={Latex[width=2mm,length=2mm]}]
	\draw[freccia] (c1.mid east) -- +(1,0) |- %
	node[-,midway,yshift=6pt,xshift=45pt,shorten <=0pt, shorten >=0pt] {\chemfig{R''-OH}}%
	node[-,midway,yshift=-13pt,xshift=45pt,shorten <=0pt, shorten >=0pt] {\chemfig{\charge{30:2pt=\chargeColor{+}}{H}}}%
	(c2.mid west);
	\draw[freccia] (c1.mid east) -- +(1,0) |- %
	node[-,midway,yshift=6pt,xshift=45pt,shorten <=0pt, shorten >=0pt] {\chemfig{HO-R''-OH}}%
	node[-,midway,yshift=-13pt,xshift=45pt,shorten <=0pt, shorten >=0pt] {\chemfig{\charge{30:2pt=\chargeColor{+}}{H}}}% 
	(c3.mid west);
	}
\end{reaction}

\section{Eteri ciclici}
Esistono eteri ciclici con anelli più grandi di quelli degli epossidi. I più comuni sono a cinque e a sei termini, tra i quali possiamo ricordare:

\begin{figure}[H]
	\centering
	\setlength{\tabcolsep}{1cm}
	\renewcommand{\arraystretch}{2}
	\begin{tabular}{ccc}
		\chemfig{[:18]*5(---O--)} & \chemfig{*6(-O-----)}   & \chemfig{*6(-O---O--)} \\
		\iupac{tetraidrofurano}   & \iupac{tetraidropirano} & \iupac{1,4-diossano}   \\
	\end{tabular}
\end{figure}

Il \ac{THF} è utilizzato come solvente sia nelle soluzioni organiche e sia nelle soluzioni acquose. É utilizzato come solvente nelle reazioni di preparazione dei reattivi di Grignard. Anche \iupac{tetraidropirano} e \iupac{1,4-diossano} sono solubili sia nell'acqua sia nei solventi organici.