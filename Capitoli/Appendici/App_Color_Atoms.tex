\chapter{Colori degli atomi nelle molecole}
\begingroup
\renewcommand{\arraystretch}{2}
\begin{table}[H]
	\centering
	\rowcolors{2}{gray!15}{}
	\begin{tabular}{cccc}
		\toprule
		\textbf{Elemento} & \textbf{Nome Colore}      & \textbf{RGB Colore} & \textbf{Colore}        \\
		\midrule
		Carbonio          & Nero\textbackslash Grigio & [144,144,144]       & \cellcolor{Carbon}     \\
		Idrogeno          & Bianco                    & [255,255,255]       & \cellcolor{Hydrogen}   \\
		Azoto             & Blu                       & [48,80,248]         & \cellcolor{Nitrogen}   \\
		Ossigeno          & Rosso                     & [255,13,13]         & \cellcolor{Oxygen}     \\
		Zolfo             & Giallo                    & [255,255,48]        & \cellcolor{Sulfur}     \\
		Fosforo           & Arancione                 & [255,128,0]         & \cellcolor{Phosphorus} \\
		Fluoro            & Verde                     & [144,224,80]        & \cellcolor{Fluorine}   \\
		Cloro             & Verde                     & [31,240,31]         & \cellcolor{Chlorine}   \\
		Bromo             & Marrone                    & [166,41,41]         & \cellcolor{Bromine}    \\
		Iodio             & Viola                     & [148,0,148]         & \cellcolor{Iodine}     \\
		\bottomrule
	\end{tabular}
\end{table}
\endgroup

