\chapter{Gruppi Funzionali}
\begingroup
\begin{center}
	\setchemfig{
		atom sep = 2em,
	}
	\rowcolors{2}{gray!15}{}
	\renewcommand{\arraystretch}{2}
	\newcommand{\molecola}[5]{\begin{XyMcompd}(#2,#3)(#4,#5){}{}#1{}\end{XyMcompd}}
	\begin{longtable}{p{0.25\textwidth} c p{0.35\textwidth}}
		\hline\rowcolor{white}
		\textbf{Classe Composto} & \textbf{Formula Generale}                                                          & \textbf{Nome del gruppo funzionale} \\
		\hline
		\endfirsthead
		\multicolumn{3}{c}%
		{{\bfseries \tablename\ \thetable{} -- Continua alla pagina precedente}}                                                                            \\
		\hline\rowcolor{white}
		\textbf{Classe Composto} & \textbf{Formula Generale}                                                          & \textbf{Nome del gruppo funzionale} \\ \hline
		\endhead
		\multicolumn{3}{r}{\textit{Continua alla pagina seguente \dots}}                                                                                    \\
		\endfoot
		\hline
		\endlastfoot

		Alcani                   & \chemfig{-C-C-}                                                                    & legame singolo                      \\
		Alcheni                  & \chemfig{-C=C-}                                                                    & legame doppio                       \\
		Alchini                  & \chemfig{-C~C-}                                                                    & legame triplo                       \\
		Idrocarburi aromatici    & \molecola{\benzenev{}}{350}{350}{230}{310}                                         & fenile                              \\
		Alogenuri alchilici      & \chemfig{R-X}                                                                      & ione alogenuro                      \\
		Alogenuri acilici        & \molecola{\tetrahedral{0==C;2==\phantom{O};4==X;1D==O}}{350}{350}{230}{310}        & acile                               \\
		Ammine primarie          & \chemfig{R-NH_2}                                                                   & amminico primario                   \\
		Ammine secondarie        & \chemfig{R-NHR}                                                                    & amminico secondario                 \\
		Ammine terziarie         & \chemfig{R-NRR'}                                                                   & amminico terziario                  \\
		Immine primarie          & \chemfig{R_2-C=NH}                                                                 & imminico primario                   \\
		Immine secondarie        & \chemfig{R_2-C=NR}                                                                 & immine secondarie                   \\
		Ammidi primarie          & \molecola{\tetrahedral{0==C;2==\phantom{O};4==NH\(_2\);1D==O}}{350}{350}{230}{310} & ammidico primario                   \\
		Ammidi secondarie        & \molecola{\tetrahedral{0==C;2==\phantom{O};4==NHR;1D==O}}{350}{350}{230}{310}      & ammidico secondario                 \\
		Ammidi terziarie         & \molecola{\tetrahedral{0==C;2==\phantom{O};4==NRR\('\);1D==O}}{350}{350}{230}{310} & ammidico terziario                  \\
		Azocomposti              & \chemfig{-N=N-}                                                                    & azo                                 \\
		Nitrili                  & \chemfig{-C~N}                                                                     & ciano                               \\
		Azoturi                  & \chemfig{-N_3}                                                                     & azido                               \\
		Nitroderivati            & \chemfig{-NO_2}                                                                    & nitro                               \\
		Nitrosocomposti          & \chemfig{-N=O}                                                                     & nitroso                             \\
		Cianati                  & \chemfig{-O-C~N}                                                                   & cianato                             \\
		Isocianati               & \chemfig{-N=C=O}                                                                   & isocianato                          \\
		Tiocianati               & \chemfig{-S-C~N}                                                                   & tiocianato                           \\
		Isotiocianati             & \chemfig{-N=C=S}                                                                   & isotiocianato                        \\
		Uretani          & \molecola{\pentamethylenei{1==\ChemForm{R_1};2==\downnobond{N}{H};4==O;5==\ChemForm{R_2}}{3D==O}}{350}{450}{300}{120} & carbammico                          \\
		Fulminati                & \chemfig{-C-N=O}                                                                   & fulminati                           \\
		Carbossile & \molecola{\tetrahedral{0==C;2==\null;1D==O;4==\null}}{200}{200}{110}{380}&Acidi carbossilici, Aldeidi, Chetoni, Esteri\\
		Acidi carbossilici & \molecola{\Rtrigonal{0==C;3D==O;1==\phantom{O};2==OH}}{200}{450}{110}{120} & carbossile\\
		Alcoli & \chemfig{-OH}&ossidrile\\
		Aldeidi & \molecola{\Rtrigonal{0==C;3D==O;1==\phantom{O};2==H}}{200}{450}{110}{120} & formile\\
		Chetoni & \molecola{\Rtrigonal{0==C;3D==O;1==\phantom{O};2==R}}{200}{450}{110}{120} & acile\\
		Esteri & \molecola{\Rtrigonal{0==C;3D==O;1==\phantom{O};2==OR}}{200}{450}{110}{120} & alcilossi\\
		Eteri & \chemfig{-O-R}&alcossi\\
		Perossidi & \chemfig{-O-O-R}&alcossi\\
		Mercaptani/tioli & \chemfig{-SH} & solfidrico\\
		Solfuri & \chemfig{-S-R} & solfidrico\\
		Solfoni & \chemfig{-SO_2-} & solfonico\\
		Acidi solfonici & \molecola{\htetrahedralS{0==S;1==R;2==OH;3D==O;4D==O}}{200}{400}{170}{200} & solfonico\\
	\end{longtable}
\end{center}
\endgroup
