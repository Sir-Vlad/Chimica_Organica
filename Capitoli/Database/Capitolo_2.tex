\chapter{Notazione di struttura}

%%%%%%%%%%%%%%%%%%%%%%%%%%%%%%%%%%%%%%%%%%%%%%%%%%%%%%%%%%%%%%%%%%%%%%%%%%%%%%%%%%%%%%%%%%%%%%%%%%%%%%%%%%%%%%%%%%%%%%%

\section{Codice SMILES}\label{sec:smiles}
Il codice \textbf{SMILES}\footnote{Per approfondire clicca qui \href{https://www.daylight.com/dayhtml/doc/theory/theory.smiles.html}{link}} (acronimo di \textit{Simplified Molecular Input Line Entry System}) è un metodo per descrivere la struttura di una molecola usando una breve stringa ASCII.

\subsection{Definizione del codice SMILES}
\subsection*{Atomi}
\addcontentsline{toc}{subsubsection}{Atomi}
Gli atomi vengono rappresentati tramite i loro simboli chimici però tra parentesi quadre, come \verb|[Au]| per l'oro. Le parentesi possono essere omesse per tutti gli atomi nel sottoinsieme organico, \verb|C, N, O, B, P, S, F, Cl, Br, I|, e che non abbiano carica, siano isotopi normali e non devono essere dei centri chirali.

Quando si usano le parentesi, viene aggiunto il simbolo \verb|H| se l'atomo tra parentesi è legato a uno o più idrogeni, seguito dal numero di atomi di idrogeno se maggiore di 1. Poi viene insita la carica con i segni \verb|+| per una carica positiva o da \verb|-| per una carica negativa, seguiti anche loro dal numero di carica se è diversa da 1. Per esempio, \ch{NH4+} viene scritto come \verb|[NH4+]|.
\subsection*{Legami chimici}
\addcontentsline{toc}{subsubsection}{Legami chimici}

\noindent I legami chimici vengono rappresentati utilizzando i seguenti simboli:
\begin{itemize}
	\item legame singolo (\verb|-|)
	\item legame doppio (\verb|=|)
	\item legame triplo (\verb|#|)
	\item legame quadruplo (\verb|$|)
	\item non legame (\verb|.|), indica che due parti non sono legate
\end{itemize}

\subsection*{Ramificazioni}
\addcontentsline{toc}{subsubsection}{Ramificazioni}
Le ramificazioni sono descritte tra parentesi tonde. Il primo atomo tra parentesi e il primo atomo dopo il gruppo tra parentesi sono entrambi legati allo stesso atomo del punto di diramazione. Il simbolo del legame deve apparire all'interno delle parentesi. Ad esempio, l'acido acetico è scritto \verb|CC(=O)O|.

L'unica forma di ramo che non richiede parentesi sono i legami ad anello, questo è dato per ridurre il numero di parentesi nella stringa.

\subsection*{Anelli}
\addcontentsline{toc}{subsubsection}{Anelli}
Le strutture ad anello vengono scritte rompendo ogni anello in un punto arbitrario per creare una struttura aciclica e aggiungendo etichette numeriche di chiusura dell'anello per mostrare la connettività tra atomi non adiacenti.
Ad esempio, il cicloesano può essere scritto come \verb|C1CCCCC1|.

Più cifre dopo un singolo atomo indicano più legami di chiusura dell'anello.  Se sono richiesti numeri di anello a due cifre, l'etichetta è preceduta da \verb|%|, quindi \verb|C%12| è un singolo legame di chiusura dell'anello di anello 12.

La scelta di un punto di interruzione dell'anello adiacente ai gruppi collegati può portare a una forma SMILES più semplice evitando le ramificazioni. Ad esempio, \iupac{cicloesan-1,2-diolo} è scritto più semplicemente come \verb|OC1CCCCC1O|; la scelta di una diversa posizione di interruzione dell'anello produce una struttura ramificata che richiede la scrittura tra parentesi.

\subsection*{Anelli aromatici}
\addcontentsline{toc}{subsubsection}{Anelli atomatici}
\noindent Gli anelli aromatici come il benzene possono essere scritti in una delle tre forme seguenti:
\begin{enumerate}
	\item In forma Kekulé con alternanza di legami singoli e doppi, ad es \verb|C1=CC=CC=C1|.
	\item Usando il simbolo del legame aromatico \verb|:|, ad esempio \verb|C1:C:C:C:C:C1|, o
	\item \label{it:smilesArom} Più comunemente, si scrivono gli atomi in forme minuscole \verb|b, c, n, o, p| e \verb|s|.
\end{enumerate}

Nel~\hyperref[it:smilesArom]{punto 3}, si presume che i legami tra due atomi siano legami aromatici. Quindi, benzene, piridina e furano possono essere rappresentati rispettivamente dagli SMILES \verb|c1ccccc1|, \verb|n1ccccc1| e \verb|o1cccc1|.

L'azoto aromatico legato all'idrogeno, come si trova nel pirrolo, deve essere rappresentato come \verb|[nH]|; quindi l'imidazolo è scritto nella notazione SMILES come \verb|n1c[nH]cc1|.

\subsection*{Stereochimica}
\addcontentsline{toc}{subsubsection}{Stereochimica}
SMILE consente, ma non richiede, la specificazione degli stereoisomeri.

La configurazione intorno ai doppi legami viene specificata utilizzando i caratteri \verb|/| e \texttt{\textbackslash}  per indicare la direzione dei legami. Ad esempio, \verb|F/C=C/F| è una rappresentazione del \iupac{\trans-1,2-difluoroetilene}, in cui gli atomi di fluoro sono sui lati opposti del doppio legame, mentre \texttt{F/C=C\textbackslash F} è una possibile rappresentazione di \iupac{\cis-1,2-difluoroetilene}, in cui i fluoro sono dalla stessa parte del doppio legame.

La configurazione al carbonio tetraedrico è specificata da \verb|@| o \verb|@@|. Considera i quattro legami nell'ordine in cui appaiono, da sinistra a destra, nella forma SMILES. Guardando verso il carbonio centrale dalla prospettiva del primo legame, gli altri tre sono in senso orario o antiorario. Questi casi sono indicati rispettivamente con \verb|@@| e \verb|@|.

Ad esempio, si consideri l'aminoacido \iupac{alanina}. Una delle sue forme SMILES è \verb|NC(C)C(=O)O|, più completamente scritta come \verb|N[CH](C)C(=O)O|. \iupac{\L-Alanina}, l'enantiomero più comune, è scritto come \verb|N[C@@H](C)C(=O)O|. Mentre \iupac{\D-Alanina} può essere scritta come \verb|N[C@H](C)C(=O)O|, perché i gruppi appaiono in senso orario.

\subsection*{Isotopi}
Gli isotopi sono specificati con un numero uguale alla massa isotopica intera che precede il simbolo atomico. Il benzene in cui un atomo è carbonio-14 è scritto come \verb|[14c]1ccccc1e| il \iupac{deuterocloroformio} è \verb|[2H]C(Cl)(Cl)Cl|.

%%%%%%%%%%%%%%%%%%%%%%%%%%%%%%%%%%%%%%%%%%%%%%%%%%%%%%%%%%%%%%%%%%%%%%%%%%%%%%%%%%%%%%%%%%%%%%%%%%%%%%%%%%%%%%%%%%%%%%%

\section{MDL Molfile}
Un MDL Molfile\footnote{Se volete approfondire, vi lascio il \attachfile[icon=Paperclip]{grafici/ctfile.pdf}file ufficiale} è un formato di file per contenere informazioni su atomi, legami, connettività e coordinate di una molecola.

Il molfile è costituito da alcune informazioni di intestazione, la tabella di connessione (CT) contenente informazioni sugli atomi, quindi sui legami e sulla tipologia di legame, seguita da sezioni che forniscono informazioni più complesse.

Il molfile è sufficientemente comune che la maggior parte, se non tutti, i sistemi/applicazioni software di cheminformatica sono in grado di leggere il formato, anche se non sempre nella stessa misura.

\lstinputlisting[caption=Esempio di file Mol]{grafici/benzene.mol}

\subsection{Descrizione molfile}
\subsubsection{Blocco intestazione}
\paragraph{Riga del titolo}\mbox{}\\
La riga del titolo contiene il nome della molecola. Può essere anche vuota ma deve esistere.
\lstinputlisting[firstline=1,lastline=1]{grafici/benzene.mol}

\paragraph{Riga timestamp programma / file}\mbox{}\\
Contiene le informazione del programma che ha generato il file e un timestamp del file.
\lstinputlisting[firstline=2,lastline=2,firstnumber=last]{grafici/benzene.mol}

\paragraph{Riga di commento}\mbox{}\\
Riga vuota che deve esistere per forza.
\lstinputlisting[firstline=3,lastline=3,firstnumber=last]{grafici/benzene.mol}

\subsubsection{Tabella dei collegamenti}
\noindent La tabella dei collegamenti è formattata nelle seguenti parti:
\begin{itemize}
	\item \textbf{Count line}: riferimenti al numero di atomi, legami e lista di atomi, chiralità e la versione Ctab.
	\lstinputlisting[firstline=4,lastline=4,firstnumber=last]{grafici/benzene.mol}
	\item \textbf{Atom block}: specifica il simbolo atomico e qualsiasi differenza di massa, carica, stereochimica e idrogeni associati per ciascun atomo.
	\lstinputlisting[firstline=5,lastline=10,firstnumber=last]{grafici/benzene.mol}
	\item \textbf{Bond block}: specifica i due atomi connessi dal legame, il tipo di legame, e se è un legame sterochimico e topologia (proprietà della catena o dell'anello) per ciascun legame.
	\lstinputlisting[firstline=11,lastline=16,firstnumber=last]{grafici/benzene.mol}
\end{itemize}

\paragraph{Count line}
Il count line è organizzato nel seguente modo:
\begin{table}[H]
	\centering
	\renewcommand{\arraystretch}{1.5}
	\begin{NiceTabular}{
		p{2cm}@{}>{\centering}p{1.7cm} >{\centering}p{1.6cm} >{\centering}p{1.6cm} >{\centering}p{2.6cm} >{\centering}p{1.6cm} >{\centering}p{1.7cm} p{1.5cm}}[hvlines]
		\rowcolor{backcolour} Valore & 6               & 6                & 0                   & 0                                                            & 0                    & 1                                                       & V2000        \\
		Descrizione                  & numero di atomi & numero di legami & numero di atom list & {\small chiralità \mbox{0 = non chirale} \mbox{1 = chirale}} & numero di voci stext & {\footnotesize numero di righe di proprietà aggiuntiv}e & versione mol \\
	\end{NiceTabular}
\end{table}

\paragraph{Specifica del atom block}
Il atom block è organizzato nel seguente modo:
\begin{table}[H]
	\centering
	\renewcommand{\arraystretch}{1.5}
	\begin{NiceTabular}{p{2cm} *{4}{p{2cm}} @{}>{\cellcolor{backcolour}}p{3cm}}[hvlines]
		\rowcolor{backcolour}
		Valore      & x                 & y                 & z                 & atomo           & \Block{2-1}{altre informazioni} \\
		Descrizione & Coordinate asse x & Coordinate asse y & Coordinate asse z & Simbolo atomico &                                 \\
	\end{NiceTabular}
\end{table}


\paragraph{Specifica del bond block}
Il bond block è organizzato nel seguente modo:
\begin{table}[H]
	\centering
	\renewcommand{\arraystretch}{1.5}
	\newcommand*{\tab}[5]{
		\begin{tabular}{*{1}{l}}
			#1 \\#2\\#3\\#4\\#5\\
		\end{tabular}
	}
	\NewDocumentCommand{\Blue}{}{\columncolor{blue!15}}
	\begin{NiceTabular}[colortbl-like]{@{}>{\Blue}ll}[hvlines]
		\rowcolor{backcolour}
		Valore              & Descrizione                                                                                                                                                                                \\
		1 Atomo             & numero dell'atomo                                                                                                                                                                          \\
		2 Atomo             & numero dell'atomo                                                                                                                                                                          \\
		Tipo di legame      & \tab{1 = Singolo, 2 = Doppio}{3 = Triplo, 4 = Aromatico}{5 = Singolo o doppio}{6 = Singolo o aromatico}{7 = Doppio o aromatico, 8 = Altro}                                                 \\
		Legame stereo       & \tab{\textbf{Legame singolo:}}{0 = non stereo, 1 = sopra, 4 = Entrambi, 6 = Sotto}{\textbf{Legame doppio:}}{0 = usare le coordinate per determinare cis o trans}{3 = cis o trans ( o entrambi) legame doppio } \\
		Non usato           &                                                                                                                                                                                            \\
		Tipologia di legame &  0 = Entrambi, 1 = Anello, 2 = Catena                                                                                                                                                                                    \\
		Stato del reagente  & \textit{vedere sul file allegato}                                                                                                                                                          \\
	\end{NiceTabular}
\end{table}

\subsubsection{Fine del file}
\noindent La fine del file viene indicata come segue:
\lstinputlisting[firstline=17,firstnumber=last]{grafici/benzene.mol}
