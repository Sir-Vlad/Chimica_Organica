\chapter{Database molecolari e software di disegno e visualizzazione}
\section{Database}
\noindent Tra i database molecolari più importanti troviamo:
\begin{itemize}
	\item \href{https://pubchem.ncbi.nlm.nih.gov}{pubchem}: database di molecole chimiche, gestito dal centro nazionale per l'Informazione biotecnologica statunitense (NCBI)
	\item \href{https://www.emolecules.com}{e-molecules}
	\item \href{https://www.ebi.ac.uk}{EMBL}: database per le sequenze nucleotidiche e proteiche
	\item \href{www.rcsb.org}{RCSB}: database di coordinate di strutture di proteine
\end{itemize}

\section{Software di disegno e visualizzatore}
\noindent Mentre tra i software più famosi abbiamo:
\begin{itemize}
	\item \href{https://www.acdlabs.com/resources/free-chemistry-software-apps/chemsketch-freeware/#chemsketch_modal}{ACDLabs ChemSketch} (Only Win): disegno di molecole e reazioni, nomenclatura, codice SMILES e InChI, visualizzazione tridimensionale, e tanto altro\dots
	\item \href{https://chemaxon.com/products/marvin}{Chemaxon MarvinSketch} (Win, Mac, Linux): \textit{non l'ho mai usato}
 \item \href{www.pymol.org}{PyMol}: visualizzatore di proteine
\end{itemize}