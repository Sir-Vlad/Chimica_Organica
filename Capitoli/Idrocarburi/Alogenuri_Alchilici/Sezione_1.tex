\section{Reazione di sostituzione nucleofila (\texorpdfstring{\mech}{SN})}
Nelle reazioni di sostituzione nucleofila, un \textbf{nucleofilo} reagisce con il \textbf{substrato} e prende il posto  del \textbf{gruppo uscente}. Queste reazioni possono essere schematizzate dalle seguenti reazioni generali:

\begin{reaction}
	\AddRxnDesc{Reazione generale \mech[2] con nucleofilo neutro}
	\chemfig{\charge{0:2pt=\:}{Nu}}
	\+{1em,1em}
	\chemfig[atom sep=15pt,bond style={white}]{R-\charge{180:2pt=\:}{L}}
	\arrow
	\chemfig[atom sep=18pt,bond style={white}]{R-\charge{180:2pt=\:,90:2pt=\chargeColor{+}}{Nu}}
	\+{1em,1em}
	\chemfig{\charge{180:2pt=\:,30:3pt=\chargeColor{-}}{L}}
\end{reaction}
\begin{reaction}
	\AddRxnDesc{Reazione generale \mech[2] con nucleofilo anionico}
	\chemfig{\charge{0:2pt=\:,30:2pt=\chargeColor{-}}{Nu}}
	\+{1em,1em}
	\chemfig[atom sep=15pt,bond style={white}]{R-\charge{180:2pt=\:}{L}}
	\arrow
	\chemfig[atom sep=18pt,bond style={white}]{R-\charge{180:2pt=\:}{Nu}}
	\+{1em,1em}
	\chemfig{\charge{180:2pt=\:,30:3pt=\chargeColor{-}}{L}}
\end{reaction}

Naturalmente queste reazioni sono teoricamente invertibili, perché il gruppo uscente ha un doppietto non condiviso che potrebbe usare per formare un legame covalente. In pratica, però, si utilizzano degli accorgimenti per costringere la reazioni ad andare solo a destra.

I meccanismi di sostituzione nucleofila sono due e si indicano con \mech[1] e \mech[2]. Entrambi i tipi dipendono dalla natura dell'elettrofilo e dell'alogenuro alchilico, dal solvente, dalla temperatura e da altri fattori.

%%%%%%%%%%%%%%%%%%%%%%%%%%%%%%%%%%%%%%%%%%%%%%%%%%%%%%%%%%%%%%%%%%%%%%%%%%%%%%%

\subsection{Meccanismo \texorpdfstring{\mech[1]}{SN1}}
Le \textbf{\mech[1]} sono \textit{reazioni di sostituzione nucleofila monomolecolare}. È classificata \textit{monomolecolare} perché nello stadio lento della reazione è coinvolta una sola molecola e quindi la reazione ha una cinetica del primo ordine.
\begin{equation*}
	\text{Velocità} = k\left[\text{Alogenoalcano}\right]
\end{equation*}
La velocità di reazione dipende solamente dalla concentrazione dell'alogenuro alchilico e non è influenzata in nessun modo dalla concentrazione del nucleofilo.

Le \mech[1] avvengono in due passaggi distinti, uno lento dove partecipa solo l'alogenuro alchilico, e consiste nella formazione del carbocatione, ed uno veloce nel quale interviene il nucleofilo.

Questo tipo di reazione avvengono facilmente quando si producono carbocationi stabili, come gli alogenuri terziari, allilici e benzilici, avvengono con più difficoltà con gli alogenuri secondari.
Lo schema generale può essere rappresentato dalla seguente equazione:
\begin{reaction}
	\setchemfig{+ sep left=1em, + sep right=1em}
	\AddRxnDesc{Meccanismo generale \mech[1]}
	\chemfig{C(-[3])(>:[:205])(<[5])(-[@{lL}0]@{L}L)}
	\arrow{<=>[lento]}
	\chemfig{@{c}\charge{0:4pt=\chargeColor{+}}{C}(-[2])(<[:-60])(>:[:-120])}
	\+ \chemfig{@{nu}\charge{180:2pt=\:}{Nu}}
	\+ \chemfig{\charge{180:2pt=\:,30:3pt=\chargeColor{-}}{L}}
	\arrow{->[veloce]}
	\chemfig{C(-[2])(>:[:210])(<[:230])(-[7]\charge{80:3pt=\chargeColor{+}}{Nu})}
	\+ \chemfig{C(-[2])(>:[:-30])(<[:-50])(-[5]\charge{80:3pt=\chargeColor{+}}{Nu})}
	\chemmove[green!60!black!70]{
		\draw[shorten <=3pt, shorten >= 3pt] (lL) .. controls +(90:.5cm) and +(115:.5cm) .. (L);
		\draw[shorten <=1pt, shorten >= 3pt] ([xshift=-4pt]nu.north west) .. controls +(90:.7cm) and +(30:1cm) .. (c);
	}
\end{reaction}

Nel primo stadio, c'è il distacco del gruppo uscente dal substrato e la formazione del carbocatione. Nel secondo stadio, il nucleofilo attacca il carbocatione su entrambe le facce. Se il substrato è chirale, nella soluzione finale si otterrà una soluzione racemica. Se il nucleofilo è una molecola neutra (come \ch{H2O} o \ch{NH3}), ci sarà un ulteriore stadio per l'eliminazione del protone dal nucleofilo.

\begin{center}
	\setchemfig{
		cram width=2pt,
		cram dash width=0.2pt,
		cram dash sep=.5pt
	}
	\begin{endiagram}[scale=1.5,debug=false,l-offset=-1,r-offset=1]
		\ENcurve{0,5[1],3[1],4[1],-2[2]}
		\ShowEa[label,label-side=left,max=all,label-pos =0.4]

		\draw[below,font=\ttfamily] (N1-1) node[xshift=.5cm] {\tiny \ch{"\chemfig{@{nu}\charge{0=\:,50:2pt=\chargeColor{-}}{Nu}}" + "\chemfig{@{c}C(-[3])(>:[:200])(<[5])(-[@{lL}0]@{L}L)}"}};
		\draw[below,font=\ttfamily] (N1-3) node {\tiny \ch{"\chemfig{@{c}\charge{0:4pt=\chargeColor{+}}{C}(-[2])(<[:-60])(>:[:-120])}"
				+ "\chemfig{@{nu}\charge{180=\:}{Nu}}"
				+ "\chemfig{\charge{180=\:,30:3pt=\chargeColor{-}}{L}}"}};
		\draw[above right,font=\ttfamily] (N1-5) node {\tiny \ch{"\chemfig{C(-[2])(>:[:210])(<[:230])(-[7]\charge{80:3pt=\chargeColor{+}}{Nu})}"
		+ "\chemfig{C(-[2])(>:[:-30])(<[:-50])(-[5]\charge{80:3pt=\chargeColor{+}}{Nu})}"}};
	\end{endiagram}
\end{center}

%%%%%%%%%%%%%%%%%%%%%%%%%%%%%%%%%%%%%%%%%%%%%%%%%%%%%%%%%%%%%%%%%%%%%%%%%%%%%%%

\subsection{Meccanismo \texorpdfstring{\mech[2]}{SN2}}
Le \textbf{\mech[2]} sono \textit{reazioni di sostituzione nucleofila bimolecolare}. È classificata \textit{bimolecolare} perché nello stadio lento della reazione sono coinvolte due molecole e quindi la reazione ha una cinetica del secondo ordine.
\begin{equation*}
	\text{Velocità} = k\left[\text{Alogenoalcano}\right]\left[\text{Nucleofilo}\right]
\end{equation*}
La velocità della reazione dipende sia dalla concentrazione del nucleofilo sia dalla concentrazione del substrato.

Le \mech[2] sono tipiche di molecole \textbf{prive di ingombro sterico} sul carbonio che regge il gruppo uscente, quindi sono tipiche degli \textbf{alogenuri metilici, primari, allilici, benzilici} e minor misura dei \textbf{secondari}, mentre non possono avvenire sui terziari, per l'eccessivo ingombro sterico.
Lo schema generale può essere rappresentato dalla seguente equazione:
\chemnameinit{}
\begin{reaction}
	\setchemfig{+ sep left=1em, + sep right=1em}
	\AddRxnDesc{Meccanismo generale \mech[2]}
	\chemfig{@{nu}\charge{0:2pt=\:,50:2pt=\chargeColor{-}}{Nu}}	\+
	\chemfig{@{c}C(-[3])(>:[:205])(<[5])(-[@{lL}0]@{L}L)}
	\arrow
	\chemname{\chemleft[\subscheme{
	\chemfig{C(-[0,1.3,,,dash pattern=on 2pt off 2pt]\charge{90:5pt=\color{blue}\(\scriptstyle\delta^-\)}{L})(-[4,1.3,,,dash pattern=on 2pt off 2pt]\charge{90:5pt=\color{blue}\(\scriptstyle\delta^-\)}{Nu})(-[2])(<[:-70])(>:[:-110])}
	}\chemright]}{stato di transizione}
	\arrow
	\chemfig{C(-[1])(>:[:-25])(<[7])(-[4]Nu)} \+ \chemfig{\charge{180:2pt=\:,30:3pt=\chargeColor{-}}{L}}
	\chemmove[green!60!black!70]{
		\draw[shorten <=0pt, shorten >= 3pt] ([xshift=4pt]nu.north east) .. controls +(90:1cm) and +(180:1cm) .. (c.west);
		\draw[shorten <=3pt, shorten >= 3pt] (lL) .. controls +(90:.5cm) and +(115:.5cm) .. (L);
	}
\end{reaction}
\chemnameinit{}

Il nucleofilo attacca il substrato dalla parte opposta a quella dove è legato il gruppo uscente. Nello stato di transizione, il carbonio si deforma per ospitare l'ingresso degli elettroni del nucleofilo che sta entrando, mentre dal lato opposto lega con il gruppo uscente. La molecola finale assume una \textbf{configurazione invertita} se il carbonio è chirale.

\begin{center}
	\setchemfig{
		cram width=2pt,
		cram dash width=0.2pt,
		cram dash sep=.5pt
	}
	\begin{endiagram}[scale=1.5,debug=false]
		\ENcurve{0,4[1],-2[2]}
		\ShowEa[label,connect={draw=none}]
		\ShowGain[label=\(\Delta H\),offset=-1,label-side=left]

		\draw[below,font=\ttfamily] (N1-1) node {\tiny \ch{"\chemfig{@{nu}\charge{0=\:,50:2pt=\chargeColor{-}}{Nu}}" +		"\chemfig{@{c}C(-[3])(>:[:200])(<[5])(-[@{lL}0]@{L}L)}"}};
		\draw[above,font=\ttfamily] (N1-2) node {\tiny \ch{!(\color{blue}Stato\;di\;Transizione)( "\chemleft[\schemestart\chemfig{C(-[0,1.3,,,dash pattern=on 2pt off 2pt]\charge{90:5pt=\color{blue}\(\scriptstyle\delta^-\)}{L})(-[4,1.3,,,dash pattern=on 2pt off 2pt]\charge{90:5pt=\color{blue}\(\scriptstyle\delta^-\)}{Nu})(-[2])(<[:-70])(>:[:-110])}\schemestop\chemright]" )}};
		\draw[above right,font=\ttfamily] (N1-3) node {\tiny \ch{"\chemfig{C(-[1])(>:[:-25])(<[7])(-[4,1.3]Nu)}" + "\chemfig{\charge{180=\:,30:3pt=\chargeColor{-}}{L}}"}};
	\end{endiagram}
\end{center}