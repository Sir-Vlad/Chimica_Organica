\chapter{Alogenuri Alchilici}
I composti che contengono un atomo di alogeno legato covalentemente con un atomo di carbonio prendono il nome di \textbf{alogenuri alchilici} (o \textbf{alogenoalcani}). Il simbolo generale per indicarli è \ch{R-X}, dove \ch{X} può essere un alogeno. Il legame \ch{C-X} è polarizzato rendendo il carbonio elettrofilo, questo fa si che può essere attaccato da un nucleofilo.

Gli alogenuri alchilici possono essere sintetizzati (per quello che sappiamo adesso) a partire da un alchene facendo reagire o con un acido alogenidrico o con un alogeno puro.
\tikzset{
arrow/.style={->,>={Latex[width=2mm,length=2mm]}}
}
\begin{center}
	\schemestart
	\chemfig{RC=CR'}
	\arrow(@c1--){0}[30,1.5]
	\chemfig{RC-CR(-[2]X)}
	\arrow(@c1--){0}[-30,1.5]
	\chemfig{RC(-[6,,2]X)(-[0]CR(-[2]X))}
	\schemestop
	\chemmove[shorten <=3pt, shorten >= 3pt]{
		\draw[arrow] (c1.east) -- +(.6,0) |- node[midway,yshift=8pt,xshift=18pt] {\chemfig{HX}} (c2.mid west);
		\draw[arrow] (c1.east) -- +(.6,0) |- node[midway,yshift=8pt,xshift=18pt] {\chemfig{X_2}} (c3.mid west);
	}
\end{center}

Le reazioni caratteristiche degli alogenuri alchilici sono: la sostituzione nucleofila e la \(\beta\)-eliminazione. Con queste reazioni gli alogenuri alchilici possono essere trasformati in alcoli, eteri, ammine, tioli e alcheni.

\subimport*{./}{Sezione_1.tex} % sostituzione nucleofila
\subimport*{./}{Sezione_2.tex} % beta-eliminazione
\subimport*{./}{Sezione_3.tex} % fattori che influenzano le reazioni di sostituzione e di eliminazione