\section{Fattori che influenzano le reazioni di sostituzione e di eliminazione}
Le reazioni di sostituzione ed eliminazione sono in competizione costante tra di loro, però ci sono dei fattori che influenzano le reazioni, che se modificati, possono dare un determinato prodotto. I fattori che influenzano le reazioni sono:
\begin{itemize}
	\item struttura molecolare del substrato
	\item basicità del nucleofilo
	\item dimensioni del nucleofilo
	\item temperatura
	\item polarità del solvente
	\item stabilità del gruppo uscente
\end{itemize}

%%%%%%%%%%%%%%%%%%%%%%%%%%%%%%%%%%%%%%%%%%%%%%%%%%%%%%%%%%%%%%%%%%%%%%%%%%%%%

\subsection*{Struttura molecolare del substrato}
\addcontentsline{toc}{subsection}{Struttura molecolare del substrato}
La \textbf{struttura molecolare} è il primo fattore che decide quale reazione può dare un alogenuro alchilico.

Le reazioni \mech[e1] ed \mech[1] sono possibili solo se la molecola è in grado di dare un carbocatione stabile e quindi sono favoriti gli alogenuri terziari, allilici e benzilici, sono meno favoriti i secondari, mentre quelli primari e metilici sono fuori discussione.

\begin{center}
	\begin{tabular}{lcr}
		\mech[e1], \mech[1] & \qquad benzilici, allilici, terziari > secondari & \qquad sono esclusi primari e metilici \\
	\end{tabular}
\end{center}

Le reazioni \mech[2], che prevedono l’attacco del nucleofilo sul carbonio che regge il gruppo uscente, sono possibili sole se non vi è ingombro sterico, quindi sono favoriti gli alogenuri metilici e primari oltre ad allilici, benzilici e vicini al carbonile, meno favoriti sono quelli secondari, mentre quelli terziari sono fuori discussione.

\begin{center}
	\begin{tabular}{lcr}
		\mech[2] & benzilici, metilici > allilici > primari > secondari & sono esclusi terziari e ingombrati \\
	\end{tabular}
\end{center}

Le reazioni \mech[e2], invece, sono poco influenzate dalla struttura molecolare perché possono avvenire su qualunque substrato purché la base sia abbastanza forte e concentrata.

%%%%%%%%%%%%%%%%%%%%%%%%%%%%%%%%%%%%%%%%%%%%%%%%%%%%%%%%%%%%%%%%%%%%%%%%%%%%%

\subsection*{Basicità del nucleofilo}
\addcontentsline{toc}{subsection}{Basicità del nucleofilo}
La \textbf{basicità del nucleofilo} è un altro fattore critico. Nucleofilo e basico sono due facce della stessa medaglia.

\textbf{Nucleofilo} è un concetto cinetico: una sostanza è molto nucleofila se attacca velocemente il carbonio in reazioni \mech[2].

\textbf{Basico} è un concetto termodinamico: una sostanza è molto basica se si lega ad \ch{H+} con grande forza e la sua \Kb\ è grande.

Una sostanza molto basica attacca l’\ch{H+} e dà eliminazione, mentre una sostanza meno basica, ma più nucleofila, attacca il carbonio in reazioni di sostituzione.

%%%%%%%%%%%%%%%%%%%%%%%%%%%%%%%%%%%%%%%%%%%%%%%%%%%%%%%%%%%%%%%%%%%%%%%%%%%%%

\subsection*{Dimensioni del nucleofilo}
\addcontentsline{toc}{subsection}{Dimensioni del nucleofilo}
Le \textbf{dimensioni del nucleofilo} sono un altro importante parametro da considerare. Il nucleofilo che attacca deve infilarsi attraverso i sostituenti per arrivare al carbonio elettrofilo. Dato che i nucleofili sono anche basi, le dimensioni diventano importanti per far prevalere un carattere o l'altro.

Un \textbf{nucleofilo basico e ingombrato} è un cattivo nucleofilo a causa dell'ingombro sterico e si comporta solo da base. Per questo motivo darà solo eliminazioni \mech[e2].

Un \textbf{nucleofilo basico di piccole dimensioni} è un buon nucleofilo e darà più facilmente sostituzioni \mech[2] sopratutto se diluito.

%%%%%%%%%%%%%%%%%%%%%%%%%%%%%%%%%%%%%%%%%%%%%%%%%%%%%%%%%%%%%%%%%%%%%%%%%%%%%

\subsection*{Temperatura}
\addcontentsline{toc}{subsection}{Temperatura}
La \textbf{temperatura} è un altro fattore importante che permette di dirigere la reazione in una direzione o l'altra.

Nelle eliminazioni aumenta il numero di molecole, infatti da due molecole se ne ottengono tre, invece, nelle sostituzioni il numero di molecole rimane costante.

Per questo motivo, le \textbf{eliminazioni} hanno un'entropia più favorevole e sono favorite ad \textbf{alte temperature}. Mentre le \textbf{sostituzioni} sono favorite \textbf{più basse temperature}.

%%%%%%%%%%%%%%%%%%%%%%%%%%%%%%%%%%%%%%%%%%%%%%%%%%%%%%%%%%%%%%%%%%%%%%%%%%%%%

\subsection*{Polarità del solvente}
\addcontentsline{toc}{subsection}{Polarità del solvente}
La \textbf{polarità del solvente} è uno degli effetti che stravolgere la tipologia di reazione che si instaura.

I \textbf{solventi polari protici} (acqua, alcol, acido acetico) solvatano bene sia cationi che anioni, quindi favoriscono la formazione dei carbocationi stabilizzandoli e rendono meno aggressivi i nucleofili favorendo le reazioni \mech[e1] e \mech[1].

I \textbf{solventi polari aprotici} (\ac{DMSO}, \ac{DMF}, acetone) solvatano bene solo i cationi. Quindi sono i solventi migliori per \mech[e2] e \mech[2], infatti, essendo aprotici, non possiedono idrogeni per creare legami idrogeno con i nucleofili e le basi.

Questo fa si che i nucleofili e le basi possono attacca con tutta la loro forza il substrato. Per rendere più liberi e più forti i nucleofili si possono usare anche cationi di grandi dimensioni come cesio, tetraetilammonio o tetrabutilammonio.

\begin{center}
	\renewcommand{\arraystretch}{2}
	\setlength{\tabcolsep}{5em} % for the horizontal padding
	\begin{tabular}{cc}
		\chemfig{H_3C-[:30]S(=[2]O)-[:-30]CH_3} & \chemfig{H-[:30]C(=[2]O)-[:-30]N(-[:30]CH_3)(-[6]CH_3)} \\
		\ac*{DMSO}                             & \ac*{DMF}                                            \\
	\end{tabular}
\end{center}

%%%%%%%%%%%%%%%%%%%%%%%%%%%%%%%%%%%%%%%%%%%%%%%%%%%%%%%%%%%%%%%%%%%%%%%%%%%%%

\subsection*{Stabilità del gruppo uscente}
\addcontentsline{toc}{subsection}{Stabilità del gruppo uscente}
La \textbf{stabilità del gruppo uscente} è un altro fattore importante perché a seconda della sua stabilità determina la velocità della reazione, perché il suo allontanamento avviene nello stadio lento di tutti i meccanismi di eliminazione e sostituzione.

I gruppi uscenti per essere considerati un buon gruppo uscente deve essere la base coniugata di un acido forte: le basi deboli sono infatti buoni gruppi uscenti.

Gli alogeni, ad eccezione del fluoro, sono quindi buoni gruppi uscenti. Gruppi come \ch{-OH} o \ch{-NH2}, sono dei cattivi gruppi uscenti e per farli uscire dalla molecola bisogna trasformali nelle loro versioni protonate.

\begin{table}[H]
	\centering
	\renewcommand{\arraystretch}{2}
	\rowcolors{2}{}{gray!15}
	\begin{tabularx}{\textwidth}{X X X X X}
		\toprule
		Tipo \mbox{di alogenuro} & \multicolumn{1}{c}{\mech[1]}                    & \multicolumn{1}{c}{\mech[2]}         & \multicolumn{1}{c}{\mech[e1]}                   & \multicolumn{1}{c}{\mech[e2]} \\
		\midrule
		Primario                 &                                                 & Altamente \mbox{favorita}             &                                                 & Avviene con basi forti        \\
		Secondario               & Può avvenire con alogenuri benzilici o allilici & Avviene in competizione con \mech[e2] & Può avvenire con alogenuri benzilici o allilici & Favorita con basi forti       \\
		Terziari                 & Favorita in solventi ossidrilici                &                                       & Avviene in competizione con \mech[1]            & Favorita con basi             \\
		\bottomrule
	\end{tabularx}
	\caption{Tabella riassuntiva delle reazioni di eliminazione e di sostituzione}
\end{table}