\section{Reazioni degli alcheni}\label{sec:rxnAlcheni}
La reazione caratteristica degli alcheni è \textbf{l'addizione al doppio legame}, questo tipo di reazione rompe il doppio legame e al suo posto si legano due nuovi sostituenti.

Le principali reazioni sono:
\begin{itemize}
	\item Idroalogenazione (\autoref{ssec:idroalogenazione})
	      \begin{reactions}
		      \AddRxnDesc{Reazione generica di Idroalogenazione degli alcheni}
		      \chemfig{C(-[3])(-[5])(=[0]C(-[1])(-[7]))} + HX \arrow \chemfig{-C(-[2])(-[6]H)-C(-[2])(-[6]X)-}
	      \end{reactions}
	\item Idratazione (\autoref{ssec:idratazione})
	      \begin{reactions}
		      \AddRxnDesc{Reazione generica di Idratazione degli alcheni}
		      \chemfig{C(-[3])(-[5])(=[0]C(-[1])(-[7]))} + \ch{H2O} \arrow \chemfig{-C(-[2])(-[6]H)-C(-[2])(-[6]OH)-}
	      \end{reactions}
	\item Idroborazione (\autoref{ssec:idroborazione})
	      \begin{reactions}
		      \AddRxnDesc{Reazione generica di Idroborazione ossidativa degli alcheni}
		      3\; \chemfig{C(-[3])(-[5])(=[0]C(-[1]H)(-[7]H))} \arrow{->[1. \ch{BH3},THF][2. \ch{H2O2}$\backslash$\ch{NaOH}]}[,2] \chemfig{-C(-[2])(-[6])-C(-[2]H)(-[6]H)-OH}
	      \end{reactions}
	\item Alogenazione (\autoref{ssec:alogenazione})
	      \begin{reactions}
		      \AddRxnDesc{Reazione generica di Alogenazione degli alcheni}
		      \chemfig{C(-[3])(-[5])(=[0]C(-[1])(-[7]))} + \ch{X2} \arrow \chemfig{-C(-[2]X)(-[6])-C(-[2])(-[6]X)-}
	      \end{reactions}
	\item Formazione di aloidrine (\autoref{ssec:aloidrine})
	      \begin{reaction}
		      \AddRxnDesc{Reazione generina di formazione di aloidrine}
		      \chemfig{C(-[3])(-[5])(=[0]C(-[1])(-[7]))} + \ch{X2}
		      \arrow{->[\ch{H2O}]}
		      \chemfig{-C(-[2,,,2]HO)(-[6])=C(-[2])(-[6]X)-}
	      \end{reaction}
	\item Riduzione (\autoref{ssec:riduzione})
	      \begin{reactions}
		      \AddRxnDesc{Reazione generica di Riduzione degli alcheni}
		      \chemfig{C(-[3])(-[5])(=[0]C(-[1])(-[7]))} + \ch{H2} \arrow \chemfig{-C(-[2])(-[6]H)-C(-[2])(-[6]H)-}
	      \end{reactions}
\end{itemize}

%%%%%%%%%%%%%%%%%%%%%%%%%%%%%%%%%%%%%%%%%%%%%%%%%%%%%%%%%%%%%%%%%%%%%%%%%%%%

\subsection{Addizione Elettrofila}
Gli elettroni \(\pi\) del doppio legame sono più esposti rispetto a quelli \(\sigma\) è essere attaccati dai reagenti. Inoltre il legame \(\pi\) è molto più debole del legame \(\sigma\) e quindi può essere coinvolto di più nelle reazioni di addizione.

I reagenti polari possono essere \textbf{elettrofili} o \textbf{nucleofili}. Gli \textit{elettrofili} sono reagenti elettron-poveri e nelle reazioni vanno alla ricerca di elettroni. I \textit{nucleofili} sono reagenti elettron-ricchi e nelle reazioni formano legami cedendo elettroni agli elettrofili. In effetti, i nucleofili sono basi di Lewis mentre gli elettrofili sono acidi di Lewis.

Il meccanismo di una addizione elettrofila avviene in due stadi. Nel primo stadio c'è la formazione del \textbf{carbocatione} per annessione di un idrogeno acido al doppio legame. Questo avviene perché il doppio legame si comporta da nucleofilo e attacca l'acido, strappando l'idrogeno.
\chemnameinit{}
\begin{reaction*}
	\chemfig{@{H}\charge{45:2pt=\chargeColor{+}}{H}} + \chemfig{C(>:[3])(<[5])=[@{dl}]C(>:[1])(<[7])} \arrow \chemname{\chemfig{C(-[2]H)(>:[:200])(<[:220])-\charge{90:4pt=\chargeColor{+}}{C}(>:[1])(<[7])}}{carbocatione}
	\chemmove[green!60!black!70]{
		\draw[shorten <=3pt, shorten >= 1pt]
		(dl).. controls +(90:1cm) and +(90:1cm) ..(H);
	}
\end{reaction*}
Nel secondo stadio, c'è l'annessione del nucleofilo sulla catena in posizione della carica positiva del carbocatione.
\chemnameinit{}
\begin{reaction*}
	\chemname{\chemfig{C(-[2]H)(>:[:200])(<[:220])-@{C+}\charge{90:4pt=\chargeColor{+}}{C}(>:[1])(<[7])}}{carbocatione} + \chemfig{[@{sb}]@{p}
	\charge{0=\:,45:5pt=\chargeColor{-}}{Nu}} \arrow \chemname{\chemfig{C(-[2]H)(>:[:200])(<[:220])-C(>:[:40])(<[:20])(-[6]Nu)}}{prodotto di addizione}
	\chemmove[green!60!black!70]{
		\draw[shorten <=3pt, shorten >= 1pt]
		([xshift=.4cm]p.250).. controls +(-90:1cm) and +(-90:1cm) ..(C+);
	}
\end{reaction*}

Poiché il primo stadio di queste reazioni di addizione è l'attacco dell'elettrone, l'intero processo si chiama \textbf{reazione di addizione elettrofila}.

%%%%%%%%%%%%%%%%%%%%%%%%%%%%%%%%%%%%%%%%%%%%%%%%%%%%%%%%%%%%%%%%%%%%%%%%%%%%

\subsection{Addizione di acidi alogenidrici}\label{ssec:idroalogenazione}
Gli acidi alogenidrici \ch{HCl}, \ch{HBr} e \ch{HI} si addizionano agli alcheni per dare alogenuri alchilici. Queste reazioni vengono condotte o con reagenti puri o in presenza di solvente polare.

Questa reazione viene effettuata in due stadi, il primo viene addizionato il protone e si rompe il doppio legame mentre il secondo si addiziona l'alogeno.

\begingroup
\setchemfig{arrow coeff=0.5,arrow offset=3pt}
\begin{reaction}
	\AddRxnDesc{Addizione di acidi alogenidrici agli alcheni}
	\chemfig{C(-[3]H_3C)(-[5]H_3C)=@{cd}C(-[1]CH_3)(-[7]CH_3)} + \chemfig{@{h}H-[@{brl}]@{br}Br}
	\arrow
	\chemfig{@{cp}\charge{90:5pt=\chargeColor{+}}{C}(-[3]H_3C)(-[5]H_3C)-C(-[2]H)(-[0]CH_3)(-[6]CH_3)} + \chemfig{@{brm}\charge{45:5pt=\chargeColor{-}}{Br}}
	\arrow
	\chemfig{C(-[2,,,2]H_3C)(-[4]H_3C)(-[6]Br)-C(-[2]H)(-[0]CH_3)(-[6]CH_3)}
	\chemmove[green!60!black!70]{
		\draw[shorten <=3pt, shorten >= 1pt] (cd) .. controls +(90:1.2cm) and +(90:1.2cm) .. (h);
		\draw[shorten <=3pt, shorten >= 1pt] (brl) .. controls +(-90:0.5cm) and +(-90:0.5cm) .. (br);
		\draw[shorten <=3pt, shorten >= 10pt] (brm) .. controls +(90:1.5cm) and +(90:1.5cm) .. (cp);
	}
\end{reaction}
\endgroup

\begin{framed}
	\textbf{NOTA}: Se dopo l'attacco, il carbonio diventa uno stereocentro bisogna scrivere tutti gli enantiomeri che si formano.
\end{framed}

%%%%%%%%%%%%%%%%%%%%%%%%%%%%%%%%%%%%%%%%%%%%%%%%%%%%%%%%%%%%%%%%%%%%%%%%%%%%

\subsection{Regola di Markovnikov e stabilità dei carbocationi}
Le \textbf{reazioni regioselettive} sono delle reazioni in cui la direzione di formazione o di rottura di un legame prevale rispetto a tutte le altre possibilità.

Queste tipologie di reazioni sono state sudiate da Markovnikov che formulò la generalizzazione, nota come \textbf{regola di Markovnikov}.

La \textbf{regola di Markovnikov} stabilisce che, nel caso di addizioni di acidi protici \ch{H―Z} ad alcheni asimmetrici, l'idrogeno dell'acido si addiziona all'atomo di carbonio del doppio legame che ha il maggior numero di atomi di idrogeno legati a sé, mentre l'alogeno si addiziona al carbonio meno idrogenato.

Questo fenomeno dipende dal fatto che quando un atomo di idrogeno si lega a uno dei due atomi di carbonio legati tramite doppio legame si forma un \textbf{carbocatione}. Un carbocatione è tanto più stabile quanto più la carica positiva può essere delocalizzata su altri atomi dello ione molecolare. Questa dispersione si realizza con uno spostamento parziale degli elettroni dai legami \(\sigma\) di \ch{C-H} e \ch{C-C} verso l'atomo di carbonio positivo. Se quest'ultimo è circondato da altri atomi di carbonio ci saranno più legami che contribuiscono a disperdere la carica.
\chemnameinit{}
\begin{reaction*}
	\chemname{\chemfig{\charge{45:3pt=\chargeColor{+}}{C}(-[2]R)(-[4]R)(-[6]R)}}{terziario}
	\quad $>$ \quad
	\chemname{\chemfig{\charge{90:3pt=\chargeColor{+}}{C}H(-[4]R)(-[6]R)}}{secondario}
	\quad $>$ \quad
	\chemname{\chemfig{R-\charge{90:3pt=\chargeColor{+}}{C}H_2}}{primario}
	\quad $>$ \quad
	\chemname{\chemfig{\charge{90:3pt=\chargeColor{+}}{C}H_3}}{metilico\\(unico)}
\end{reaction*}

%%%%%%%%%%%%%%%%%%%%%%%%%%%%%%%%%%%%%%%%%%%%%%%%%%%%%%%%%%%%%%%%%%%%%%%%%%%%

\subsection{Addizione di acqua - Idratazione catalizzata da acidi}\label{ssec:idratazione}
L'addizione di acqua è chiamata \textbf{idratazione}. Generalmente, questa reazione avviene in presenza di catalizzatore acido (\ch{H2SO4} concentrato), l'acqua si addiziona al doppio legame di un alchene per formare un alcol.

\paragraph{Meccanismo di reazione}
\begin{reaction}
	\AddRxnDesc{Idratazione degli alcheni}
	\chemfig{CH_3CH=[@{ch}]CH_2} + \chemfig{@{op}\charge{45:5pt=\chargeColor{+},90:3pt=\:}{O}(-[0]H)(-[@{h1}4]@{h}H)(-[6]H)}
	\arrow(.base east--.base west){<=>}
	\chemfig{CH_3@{cp}\charge{90:3pt=\chargeColor{+}}{C}HCH_3} \+{.5em,.5em} \chemfig{@{o}\charge{90:1pt=\:,180:1pt=\:}{O}(-[0]H)(-[6]H)}
	\arrow(@c2--){<=>}[-90]
	\chemfig{CH_3@{cn}CHCH_3(-[6,,3]@{oc}\charge{45:2pt=\chargeColor{+},-90:2pt=\:}{O}(-[5]H)(-[@{lho}7]@{hop}H))} + \chemfig{H-@{oh}\charge{-90:2pt=\:,90:2pt=\:}{O}-H}
	\arrow{<=>}[-180]
	\chemfig{CH_3CHCH_3(-[6,,3]\charge{-90:2pt=\:,180:2pt=\:}{O}H)} + \chemfig{\charge{45:5pt=\chargeColor{+},90:3pt=\:}{O}(-[0]H)(-[4]H)(-[6]H)}
	\chemmove[green!60!black!70]{
		\draw[shorten <=3pt, shorten >= 1pt] (ch) .. controls +(90:.8cm) and +(90:.8cm) .. (h);
		\draw[shorten <=3pt, shorten >= 1pt] (h1).. controls +(-120:.5cm) and +(-135:.5cm) .. (op);
		\draw[shorten <=3pt, shorten >= 1pt] ([xshift=-3pt]o.150).. controls +(-120:1cm) and +(-90:1cm) .. (cp);
		\draw[shorten <=3pt, shorten >= 6pt] (oh).. controls +(-90:1cm) and +(0:1cm) .. (hop);
		\draw[shorten <=3pt, shorten >= 1pt] (lho).. controls +(45:.2cm) and +(0:.3cm) .. (oc.east);
	}
\end{reaction}

%%%%%%%%%%%%%%%%%%%%%%%%%%%%%%%%%%%%%%%%%%%%%%%%%%%%%%%%%%%%%%%%%%%%%%%%%%%%

\subsection{Idroborazione ossidativa}\label{ssec:idroborazione} 
L'\textbf{idroborazione} è una reazione in cui il borano \ch{BH3} si addiziona al doppio legame di un alchene, in solvente THF. L'idroborazione ha due caratteristiche principali:
\begin{itemize}
	\item è una reazione \textbf{anti-Markovnikov}
	\item è \textbf{stereospecifica}
\end{itemize}
Viene definita reazione \textbf{anti-Markovnikov} perché l'atomo di boro si leghi all'atomo di carbonio del doppio legame meno sostituito e l'idrogeno si leghi all'atomo di carbonio più sostituito, mentre è \textbf{stereospecifica} perché porta l'idrogeno e il boro a legarsi dallo stesso lato del piano dell'alchene (\textbf{sin-addizione}). Inoltre, nell'idroborazione non si verificano fenomeni di trasposizione o riarraggiamenti.

\begin{reaction}
	\AddRxnDesc{Idroborazione ossidativa}
	\chemfig{-[:30]@{C1}=_[@{dl}:-30]} \+ \chemfig{@{B}B(-[2]H)(-[:-30]H)(-[@{H1}:210]H)}
	\arrow
	\chemfig{-[:30]-[:-30]-[:30]BH_2}
	\arrow{->[\chemfig{-[:30]=_[:-30]}]}[,1.3]
	\chemfig{-[:30]-[:-30]-[:30]\chemabove{B}{H}-[:-30]-[:30]-[:-30]}
	\arrow{->[*{0}\chemfig{-[:30]=_[:-30]}]}[-90,1.3]
	\chemfig{-[:30]-[:-30]-[:30]B(-[2]-[:30]-[2])-[:-30]-[:30]-[:-30]}
	\arrow(.mid west--){->[\chemfig{H_2O_2}][\chemfig{NaOH}]}[-180,1.3]
	3 \chemfig{-[:30]-[:-30]-[:30]OH}
	% - arrow
	\chemmove[green!60!black!70]{
		\draw[shorten <=3pt, shorten >= 1pt] (dl).. controls +(225:1cm) and +(270:1cm) .. (B);
		\draw[shorten <=3pt, shorten >= 1pt] (H1).. controls +(120:0.7cm) and +(90:0.7cm) .. (C1);
	}
\end{reaction}

%%%%%%%%%%%%%%%%%%%%%%%%%%%%%%%%%%%%%%%%%%%%%%%%%%%%%%%%%%%%%%%%%%%%%%%%%%%%

\subsection{Addizione di Alogeni}\label{ssec:alogenazione}
Il cloro e il bromo reagiscono con gli alcheni a temperatura ambiente addizionando atomi di alogeno ai due carboni del doppio legame.
\begin{reaction*}
	\chemfig{CH_3CH=CHCH_3} + \ch{Br2} \arrow(--.193){->[\ch{CH2Cl2}]}[,1.3] \chemfig{CH_3CH(-[2,,3]Br)-CHCH_3(-[2]Br)}
\end{reaction*}
Anche il fluoro si addiziona con gli alcheni ma la reazione è troppo veloce e difficile da controllare mentre lo iodio si addiziona ma con tempi troppo lunghi.

L'addizione di alogeni ai cicloalcheni può dare, teoricamente, sia isomeri \cis\;che isomeri \trans, però effettivamente si osserva solo l'isomero \trans. Questo tipo di reazioni dove si osserva solo un tipo di isomero si chiamano \textbf{reazione stereoselettive}.

\paragraph{Meccanismo di reazione}\mbox{}\\
Il doppio legame dell'alchene attacca \ch{X2} (legando con un atomo di alogeno e espellendo l'altro), il carbocatione risultante è molto instabile e per stabilizzarsi accetta gli elettroni dell'alogeno formando un ciclo a tre termini che viene chiamato \textbf{intermedio a ponte alonio}.

\begingroup
\chemnameinit{}
\begin{reaction}
	\AddRxnDesc{Addizione di alogeni agli alcheni (stadio 1)}
	\chemfig{C(-[3]CH_3)(-[5]H)(=[@{dl}]C(-[1]CH_3)(-[7]H))} + \chemfig{@{x1}X-[@{lx}]@{x2}X}
	\arrow[,0.8]
	\chemfig{@{c}\charge{45:3pt=\chargeColor{+}}{C}(-[3]CH_3)(-[5]H)(-C(-CH_3)(-[2]@{x}\charge{180=\:}{X})(-[6]H))}
	\arrow(--.base west)[,0.8]
	\chemname{\chemfig{[:90]C*3((<[:-55]H)(<:[:-10]CH_3)-[3]\charge{90:3pt=\chargeColor{+}}{X}-C(<:[:190]H_3C)(<[:235]H)-)}}{Intermedio a\\ponte alonio}
	\chemmove[green!60!black!70]{
		\draw[shorten <=3pt, shorten >= 1pt] (dl).. controls +(90:1.5cm) and +(90:1cm) .. (x1);
		\draw[shorten <=3pt, shorten >= 1pt] (lx).. controls +(90:.8cm) and +(90:.8cm) .. (x2);
		\draw[shorten <=3pt, shorten >= 5pt] (x).. controls +(180:.8cm) and +(90:.8cm) .. (c);
	}
\end{reaction}
\endgroup

L'intermedio, che si forma, è molto più stabile di un normale carbocatione perché la carica si delocalizza su tutto l'anello. Il ponte alonio impedisce alla molecola di ruotare su se stessa e per questo motivo, costringe l'alogeno ad attaccare il carbocatione dall'altra faccia (posizione anti). Come solvente si utilizza etere o \ch{CCl4} per costringere l'alogeno il ponte alonio.

\begingroup
\chemnameinit{}
\begin{reaction}
	\AddRxnDesc{Addizione di alogeni agli alcheni (stadio 2)}
	\chemname[15pt]{\chemfig{[:90]@{ca1}C*3((<[:-55]H)(<:[:-10]CH_3)-[@{lX}]@{X}\charge{90:3pt=\chargeColor{+}}{X}-@{ca2}C(<:[:190]H_3C)(<[:235]H)-)}}{attacco anti} \+ \chemfig{@{x}\charge{180=\:,45:3pt=\chargeColor{-}}{X}}
	\arrow(--.mid west)
	\chemname{\chemfig{C(<:[:195]H_3C)(-[3]X)(<[5]H)(-C(-[7]X)(<[:15]H)(<:[1]CH_3))}}{a}
	\+ 
	\chemname{\chemfig{C(<:[3]H_3C)(-[5]X)(<[:165]H)(-C(-[1]X)(<[7]H)(<:[:-15]CH_3))}}{b} 
	\chemmove[green!60!black!70]{
		\draw[shorten <=3pt, shorten >= 1pt] (x.mid west).. controls +(-90:1.5cm) and +(-90:1.5cm) .. (ca1) node[xshift=-5pt,yshift=-15pt] {\tiny a};
		\draw[shorten <=3pt, shorten >= 1pt] (x.south west).. controls +(-90:1.5cm) and +(-90:1.5cm) .. (ca2) node[xshift=5pt,yshift=-15pt] {\tiny b};
		% \draw[shorten <=3pt, shorten >= 1pt] (lX).. controls +(45:.3cm) and +(0:.4cm) .. (X);
	}
\end{reaction}
\endgroup
%%%%%%%%%%%%%%%%%%%%%%%%%%%%%%%%%%%%%%%%%%%%%%%%%%%%%%%%%%%%%%%%%%%%%%%%%%%%

\subsection{Formazione di aloidrine}\label{ssec:aloidrine}
La formazione di aloidrine avviene quando l'addizione di alogeni avviene in presenza di piccola quantità di acqua. La reazione è uguale alla reazione con alogeni tranne per il secondo stadio dove l'acqua è più veloce ad attaccare lo ione alonio rispetto all'alogeno.

\paragraph{Meccanismo di reazione}\mbox{}\\
\begingroup
\chemnameinit{}
\begin{reaction}
	\AddRxnDesc{Addizione di alogeni agli alcheni (stadio 2)}
	\chemname[15pt]{\chemfig{[:90]@{ca1}C*3((<[:-55]H)(<:[:-10]CH_3)-[@{lX}]@{X}\charge{90:3pt=\chargeColor{+}}{X}-@{ca2}C(<:[:190]H_3C)(<[:235]H)-)}}{attacco anti} \+ \chemfig{@{x}\charge{90=\:,180=\:,270=\:,135:3pt=\chargeColor{-}}{O}H}
	\arrow(--.mid west)
	\chemname{\chemfig{C(<:[:195]H_3C)(-[3]X)(<[5]H)(-C(-[7]OH)(<[:15]H)(<:[1]CH_3))}}{a}
	\+
	\chemname{\chemfig{C(<:[3]H_3C)(-[5]OH)(<[:165]H)(-C(-[1]X)(<[7]H)(<:[:-15]CH_3))}}{b}
	\chemmove[green!60!black!70]{
		\draw[shorten <=3pt, shorten >= 1pt] (x.mid west).. controls +(-90:1.5cm) and +(-90:1.5cm) .. (ca1) node[xshift=-5pt,yshift=-15pt] {\tiny a};
		\draw[shorten <=3pt, shorten >= 1pt] (x.south west).. controls +(-90:1.5cm) and +(-90:1.5cm) .. (ca2) node[xshift=5pt,yshift=-15pt] {\tiny b};
		% \draw[shorten <=3pt, shorten >= 1pt] (lX).. controls +(45:.3cm) and +(0:.4cm) .. (X);
	}
\end{reaction}
\endgroup

%%%%%%%%%%%%%%%%%%%%%%%%%%%%%%%%%%%%%%%%%%%%%%%%%%%%%%%%%%%%%%%%%%%%%%%%%%%%

\subsection{Riduzione degli alcheni}\label{ssec:riduzione}
La riduzione degli alcheni, chiamata più comunemente \textbf{idrogenazione} è un processo chimico dove tramite catalizzatore metallico, generalmente platino disperso in polvere di carbonio indicato con \ch{Pt\textbackslash C}, viene addizionato idrogeno gassoso al doppio legame per dare alcani. L'addizione dell'idrogeno avviene dallo stesso lato del piano dell'alchene.

\begingroup
\setchemfig{
	% arrow offset=3em,
}
\begin{reaction}
	\AddRxnDesc{Riduzione degli alcheni}
	\chemname{\chemfig{C(-[:120]H_3C)(-[:-120]Cl)(=C(-[:60]Cl)(-[:-60]CH_3))}}{\iupac{\trans-2,3-dicloro-2-butene}}
	\arrow(.mid east--.mid west){->[\ch{H2}][\ch{Pt\textbackslash C}]}[,1.3]
	\chemname{\chemfig{C(-[:120]H)(<:[:200]H_3C)(<[:-120]Cl)(-C(-[:60]H)(<:[:-20]Cl)(<[:-60]CH_3))}}{\iupac{\cip{2R,3R}-2,3-diclorobutano}}
	\+{2em,2em}
	\chemname{\chemfig{C(-[:-120]H)(<:[:120]H_3C)(<[:160]Cl)(-C(-[:-60]H)(<:[:60]Cl)(<[:20]CH_3))}}{\iupac{\cip{2S,3S}-2,3-diclorobutano}}
\end{reaction}
\endgroup