\section{Struttura degli alcheni}\label{sec:strAlcheni}
Gli \textbf{alcheni} sono idrocarburi insaturi che contengono almeno un doppio legame e hanno formula generale \ch{C_{n}H_{2n}}. Il capostipite degli alcheni è l'etene (\ch{H2C=CH2}).

I composti che contengono due legami doppi si chiamano \textbf{dieni}, ma esistono anche molecole con un numero variabili di doppi legami. A seconda di come posizionati i doppi legami possiamo avere:
\begin{itemize}
	\item \textbf{legami doppi cumulati}, quando si trovano in successione
	\item \textbf{legami doppi coniugati}, quando si trovano alternanti
	\item \textbf{legami doppi isolati}, quando c'è più di un legame singolo tra i legami doppi
\end{itemize}

\begin{center}
	\ch{!(\color{blue}cumulati)( C=C=C )} \qquad \ch{!(\color{blue}coniugati)( C=C-C=C )} \qquad \ch{!(\color{blue}isolati)( C=C-C-C=C )}
\end{center}

Ad oggi, diversi alcheni, come l'etene e il propilene, rivestono un'importanza fondamentale nell'industria chimica come base per produrre molte sostanze organiche. Per questo motivo e anche perché in natura si trova in quantità bassissime, si produce per cracking termico dall'etano:

\begin{reactions}
	\AddRxnDesc{Cracking termico dell'etano in etilene}
	\ch{ !(Etano)( CH3CH3 ) ->[800 - 900 °C][(cracking\;termico)] !(Etilene)( CH2=CH2 ) + H2}
\end{reactions}

\subsection{Caratteristiche del doppio legame}
I carbonio interessati nel doppio legame hanno ibridazione \(sp^2\) e sono legati a due sostituenti ciascuno. La loro geometria è planare con un angolo di legame di \ang{120} e una lunghezza di 1,34\angstrom.

Possiamo immagine la formazione del doppio legame a due stadi, il primo stadio si forma il legame \(\sigma\) per sovrapposizione frontale degli orbitali mentre il secondo stadio si forma il legame \(\pi\) per sovrapposizione laterale degli orbitali.

La rotazione intorno al doppio legame è impedita a causa dell'energia necessaria per rompere il legame \(\pi\) a temperatura ambiente.

