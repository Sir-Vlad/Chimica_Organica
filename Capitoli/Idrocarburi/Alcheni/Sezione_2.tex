\section{Isomeria \cis\texorpdfstring{\,\!}{} - \trans\texorpdfstring{\,\!}{} nei alcheni}
Dal momento che la rotazione intorno al doppio legame è impedita, gli alcheni che hanno entrambi gli atomi sostituiti presentano isomeria \cis-\trans.

Dal momento che l'isomeria \cis-\trans\,può ricoprire il suo ruolo solo quando abbiamo i carboni bisostituiti. Ma cosa fare quando sono tri- o tetrasostituiti? In questo caso si utilizza il \textbf{sistema E-Z}.
Questo sistema assegna una priorità a ogni sostituente e se i gruppi ad alta priorità si trovano dalla stessa parte si utilizza \Z mentre se i gruppi ad alta priorità si trovano in posizione opposta si utilizza \E.

Il primo passo, per assegnare la configurazione \E o \Z ad un doppio legame è quello di assegnare una priorità a ogni sostituente.

\paragraph{Regola di priorità}\mbox{}\label{sec:prioritaConfigurazione}
\newcommand*{\sopra}[2]{\stackText{\stackon[5pt]{#1}{\scriptsize\color{blue}#2}}}
\begin{enumerate}
	\item \label{it:priorità_1} La priorità è basata sul numero atomico: più alto è il numero atomico, più è alta la priorità.
	      \begin{center}
		      \sopra{I}{53} > \sopra{Br}{35} > \sopra{Cl}{17} > \sopra{S}{16} > \sopra{F}{9} > \sopra{O}{8} > \sopra{N}{7} > \sopra{C}{6} > \sopra{H}{1}
	      \end{center}
	\item Se non si può assegnare la priorità ai primi atomi dei sostituenti, si passa ai successivi fino a quando non si trova una differenza. Una volta trovata si assegna la priorità come nel~\autoref{it:priorità_1}.
	\item Gli atomi legati attraverso legami multipli sono equivalenti allo stesso numero di atomi legati attraverso legami singoli.
	\begin{reaction*}
		\chemfig[atom sep=2em]{-CH=CH_2} \arrow{->[\scriptsize trattato][\scriptsize come]} \chemfig[atom sep=2em]{-CH(-[2]C)(-[0]CH_2(-[2]C))}
	\end{reaction*}

	\begin{reaction*}
		\chemfig[atom sep=2em]{-CH(=[2]O)-} \arrow{->[\scriptsize trattato][\scriptsize come]} \chemfig[atom sep=2em]{-C(-[2]O(-[0]C))(-[0]O)(-[6]H)}
	\end{reaction*}
\end{enumerate}