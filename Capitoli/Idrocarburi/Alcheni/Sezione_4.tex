\section{I dieni}
Come abbiamo accennato nella \autoref{sec:strAlcheni}, i dieni sono alcheni con due legami doppi e possono essere coniugati o isolati. I composti coniugati sono molto importanti perché il legame singolo tra i due legami doppi ha un parziale carattere di doppio legame. Questo permette la delocalizzazione degli elettroni su tutta la molecola rendendola più stabile tramite le forme di risonanza.

\begin{figure}[H]
	\begin{center}
		\schemestart
		\chemleft[
			\subscheme{
				\chemfig{@{c1n}\charge{135:1pt=\chargeColor{-},90:2pt=\:}{C}H_2-[@{el1}]CH=[@{dl1}]CH-[@{el2}]\charge{90:3pt=\chargeColor{+}}{C}H_2}
				\arrow{<->}[,0.8]
				\chemfig{CH_2=[@{dl2}]CH-[@{sl1}]CH=[@{dl3}]@{cf}CH_2}
				\arrow{<->}[,0.8]
				\chemfig{\charge{90:3pt=\chargeColor{+}}{C}H_2-CH=CH-\charge{80:7pt=\chargeColor{-},90:2pt=\:}{C}H_2}
				}
			\chemright]
			\chemmove{
				\draw[shorten <=3pt, shorten >= 1pt] (c1n).. controls +(90:.5cm) and +(90:.5cm) .. (el1);
				\draw[shorten <=3pt, shorten >= 1pt] (dl1).. controls +(-90:.5cm) and +(-90:.5cm) .. (el2);
				\draw[shorten <=3pt, shorten >= 1pt] (dl2).. controls +(90:.5cm) and +(90:.5cm) .. (sl1);
				\draw[shorten <=3pt, shorten >= 1pt] (dl3).. controls +(90:.5cm) and +(90:.5cm) .. (cf);
				}
		\schemestop
	\end{center}
	\caption{Forme di risonanza del \iupac{1,3-butadiene}}
\end{figure}

L'ibrido di risonanza è la media pesata di tutte le forme di risonanza e la miglior struttura è quella che contiene più legami e meno cariche.