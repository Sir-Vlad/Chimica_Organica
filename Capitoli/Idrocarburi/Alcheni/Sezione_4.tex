\section{I dieni}
Come abbiamo accennato nella \autoref{sec:strAlcheni}, i dieni sono alcheni con due legami doppi e possono essere coniugati o isolati. I composti coniugati sono molto importanti perché il legame singolo tra i due legami doppi ha un parziale carattere di doppio legame. Questo permette la delocalizzazione degli elettroni su tutta la molecola rendendola più stabile tramite le forme di risonanza.

\begin{figure}[H]
	\begin{center}
		\schemestart
		\chemleft[
		\subscheme{
		\chemfig{@{c1n}\charge{270:2pt=\chargeColor{-},90:1pt=\:}{}-[@{el1}:30]=_[@{dl1}:-30]-[@{el2}:30]\charge{90:3pt=\chargeColor{+}}{}}
		\arrow{<->}[,0.8]
		\chemfig{=_[@{dl2}:30]-[@{sl1}:-30]=_[@{dl3}:30]@{cf}}
		\arrow{<->}[,0.8]
		\chemfig{\charge{90:3pt=\chargeColor{+}}{}-[:30]=_[:-30]-[:30]\charge{90:5pt=\chargeColor{-},90:1pt=\:}{}}
		}
		\chemright]
		\chemmove{
			\draw[shorten <=4pt, shorten >= 2pt] (c1n).. controls +(90:.3cm) and +(120:.4cm) .. (el1);
			\draw[shorten <=2pt, shorten >= 2pt] (dl1).. controls +(45:.3cm) and +(135:.3cm) .. (el2);
			\draw[shorten <=3pt, shorten >= 1pt] (dl2).. controls +(-45:.3cm) and +(-135:.3cm) .. (sl1);
			\draw[shorten <=3pt, shorten >= 1pt] (dl3).. controls +(90:.5cm) and +(90:.5cm) .. (cf);
		}
		\schemestop
	\end{center}
	\caption{Forme di risonanza del \iupac{1,3-butadiene}}
\end{figure}

L'ibrido di risonanza è la media pesata di tutte le forme di risonanza e la miglior struttura è quella che contiene più legami e meno cariche.