\chapter{Alchini}\label{chp:Alchini}

\section{Struttura degli alchini}
Gli alchini (o acetileni) sono idrocarburi aventi almeno un triplo legame carbonio-carbonio. Il carbonio del triplo legame ha ibridazione \(sp\) e per questo ha geometria lineare con un angolo di legame di \ang{180} e una lunghezza di 1,21\angstrom. Avendo geometria lineare non avranno isomeria geometrica.

Le reazioni tipiche che vanno incontro gli alchini sono uguali alle reazioni degli alcheni viste nel \autoref{sec:rxnAlcheni}.

\section{Acidità degli alchini}
L'idrogeno del triplo legame è debolmente acido quindi può essere strappato da una base forte. Ad esempio, la sodio ammide è in grado di trasformare gli alchini in acetiluri.

\begin{reaction}
	\AddRxnDesc{Reazione acido-base alchini}
	\ch{R-C+C-H} + \ch{!(Sodio\;Ammide)( Na+ NH2- )} \arrow{->[\tiny \ch{NH3} liquida]} \ch{!(Acetiluro\;di\;Sodio)( R-C+C- Na+ )} + \ch{NH3}
\end{reaction}

Questo tipo di reazione avviene perché più ibridazione del carbonio ha crescente carattere di tipo \(s\) e crescente carattere di tipo \(p\), più l'acidità dell'idrogeno a esso legato cresce.