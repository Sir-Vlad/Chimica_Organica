\section{Disostituzione e polisostituzione}
Nella \ac{SEA} di un benzene monosostituito, il sostituente già presente sull'anello influenza la reazione determinando la posizione dei nuovi sostituenti e la velocità della sostituzione.

Certi sostituenti orientano un secondo gruppo entrante favorendo le posizioni orto e para mentre altri preferiscono la posizione meta. Detto questo possiamo classificare i sostituenti come \textbf{orto-para orientati} e \textbf{meta orientati}.

Possiamo fare lo stesso discorso con la velocità e classificare i sostituenti come \textbf{attivanti}, ovvero quelli che aumentano la velocità di reazione e \textbf{disattivanti}, cioè quelli che diminuiscono la velocità di reazione.

Nella~\autoref{tab:AttEDisatt} sono elencati gli effetti orientanti e attivanti-disattivanti per i gruppi funzionali più comuni.

\begingroup
\setchemfig{double bond sep=2pt}
\renewcommand{\arraystretch}{2}
\scriptsize
\begin{table}[htb]
	\centering
	\begin{NiceTabular}{|c|p{2.9cm}|llllll|}
		\hline
		\Block{4-1}<\rotate>{Orto - para orientati} & Fortemente \mbox{attivanti}    & \chemfig{-[,0.5]\charge{90:2pt=\:}{N}H_2}                  & \chemfig{-[,0.5]\charge{90:2pt=\:}{N}HR}                    & \chemfig{-[,0.5]\charge{90:2pt=\:}{N}HR_2}                    & \chemfig{-[,0.5]\charge{90:2pt=\:,270:2pt=\:}{O}H}             & \chemfig{-[,0.5]\charge{90:2pt=\:,270:2pt=\:}{O}R} &                                    \\
		                                            & Moderatamente \mbox{attivanti} & \chemfig{-[,0.5]\charge{90:2pt=\:}{N}HCR(=[2,,3]O)}        & \chemfig{-[,0.5]\charge{90:2pt=\:}{N}HCAc(=[2,,3]O)}        & \chemfig{-[,0.5]\charge{90:2pt=\:,270:2pt=\:}{O}CR(=[2,,2]O)} & \chemfig{-[,0.5]\charge{90:2pt=\:,270:2pt=\:}{O}CAc(=[2,,2]O)} &                                                    &                                    \\
		                                            & Debolmente \mbox{attivanti}    & \chemfig{-[,0.5]R}                                         & \chemfig[atom sep=1.5em]{-[,0.5]*6(-=-=-=)}                 &                                                               &                                                                &                                                    &                                    \\
		                                            & Debolmente \mbox{disattivanti} & \chemfig{-[,0.5]\charge{0:2pt=\:,90:2pt=\:,270:2pt=\:}{F}} & \chemfig{-[,0.5]\charge{0:2pt=\:,90:2pt=\:,270:2pt=\:}{Cl}} & \chemfig{-[,0.5]\charge{0:2pt=\:,90:2pt=\:,270:2pt=\:}{Br}}   & \chemfig{-[,0.5]\charge{0:2pt=\:,90:2pt=\:,270:2pt=\:}{I}}     &                                                    &                                    \\
		\hline
		\Block{2-1}<\rotate>{Meta orientati}        & Moderatamente disattivanti     & \chemfig{-[,0.5]CH(=[2]O)}                                 & \chemfig{-[,0.5]CR(=[2]O)}                                  & \chemfig{-[,0.5]COH(=[2]O)}                                   & \chemfig{-[,0.5]COR(=[2]O)}                                    & \chemfig{-[,0.5]CNH_2(=[2]O)}                      & \chemfig{-[,0.5]SOH(=[2]O)(=[6]O)} \\
		                                            & Fortemente \mbox{disattivanti} & \chemfig{-[,0.5]NO_2}                                      & \chemfig{-[,0.8]N\charge{45:3pt=\chargeColor{+}}{H}_3}      & \chemfig{-[,0.5]CF_3}                                         & \chemfig{-[,0.5]CCl_3}                                         &                                                    &                                    \\
		\hline
	\end{NiceTabular}
	\caption{Effetto dei sostituenti su un'altra SEA}\label{tab:AttEDisatt}
\end{table}
\endgroup

\noindent Se paragoniamo questi gruppi possiamo fare le seguenti generalizzazioni:
\begin{enumerate}
	\item\label{en:CAs5.4.1} I gruppi alchilici, i gruppi fenilici e i sostituenti nei quali l'atomo all'anello ha una coppia non condivisa di elettroni sono orto - para orientanti. Tutti gli altri sono meta orientanti
	\item\label{en:CAs5.4.2} Tutti i gruppi orto-para orientanti sono attivanti, eccetto gli alogeni che sono disattivanti
	\item\label{en:CAs5.4.3} Tutti i gruppi meta orientanti hanno carica positiva, parziale o intera, sull'atomo legato all'anello
\end{enumerate}

Per capire perché si possono fare le generalizzazioni precedenti, prendiamo come esempio la clorurazione del fenolo, per spiegare il~\autoref{en:CAs5.4.1}.
\begin{figure}[H]
	\centering
	\tikzset{%
	>={Latex[width=2mm,length=2mm]}}
	\begin{tikzpicture}[node distance=3cm,align=center]
		\node (start)[]{\scriptsize\schemestart\chemfig{*6(-=-=(-OH)-=)} \arrow(.-25--){0}[,0]\+ \ch{Cl2}\schemestop};
		\node (para) [right of=start,xshift=5.5cm] {\paraFenolo};
		\node (orto) [above of=para,xshift=1.4cm] {\ortoFenolo};
		\node (meta) [below of=para,xshift=-.5cm] {\metaFenolo};


		\draw[->] (start) -- node[midway,yshift=10pt,xshift=12pt] {para} (para);
		\draw[->] (start) -- +(2,0) |- node[near end,yshift=10pt] {orto} (orto);
		\draw[->] (start) -- +(2,0) |- node[near end,yshift=10pt] {meta} (meta);
	\end{tikzpicture}
\end{figure}

Nel caso dell'attacco in orto e in para, in una delle forme limite dell'intermedio di reazione, la carica positiva si colloca sul carbonio che porta all'ossidrile. Lo spostamento degli elettroni sull'ossigeno delocalizza ulteriormente la carica, aumentando la stabilità dell'intermedio.

Possiamo concludere che i sostituenti che possiedono elettroni non condivisi sull'atomo direttamente collegato all'anello sono orto-para orientanti.

Allo stesso modo, per spiegare l'effetto meta orientante prendiamo come esempio la nitrazione del nitrobenzene.

\begin{figure}[H]
	\centering
	\tikzset{%
	>={Latex[width=2mm,length=2mm]}}
	\begin{tikzpicture}[node distance=3.5cm,align=center]
		\node (start)[]{\scriptsize\schemestart\chemfig{*6(-=-=(-\chemabove{N}{\color{red}\scriptstyle\oplus}(-[1]\charge{45:3pt=\chargeColor{-}}{O})(=[3]O))-=)} \arrow{0}[,0]\+ \ch{NO2}\schemestop};
		\node (para) [right of=start,xshift=5.5cm] {\paraNitroBenzene};
		\node (orto) [above of=para,xshift=1cm] {\ortoNitroBenzene};
		\node (meta) [below of=para,xshift=0.5cm] {\metaNitroBenzene};


		\draw[->] (start) -- node[midway,yshift=10pt,xshift=12pt] {para} (para);
		\draw[->] (start) -- +(2,0) |- node[near end,yshift=10pt] {orto} (orto);
		\draw[->] (start) -- +(2,0) |- node[near end,yshift=10pt] {meta} (meta);
	\end{tikzpicture}
\end{figure}

In questo caso, gli attacchi orto e para non si osservano perché in una forma limite dell'ibrido di risonanza ha due cariche adiacenti e questa situazione rende estremamente sfavorevole osservazione mentre nel'attacco meta tutto ciò non si verifica e di conseguenza è quello preferito.

Adesso, per spiegare il~\autoref{en:CAs5.4.2} possiamo dire che i sostituenti meta orientanti attirano a se gli elettroni dell'anello, perché portano su di loro cariche positive parziali o totali. Questo fa si che attirando gli elettroni dell'anello su di loro, disattivano l'anello stesso. Al contrario, i gruppi orto-para orientanti possono fornire elettroni all'anello e perciò attivano l'anello. L'unica eccezione tra i sostituenti sono gli alogeni che essendo fortemente elettron-attrattori disattivano l'anello ma avendo doppietti di non legame li rende orto-para orientanti.