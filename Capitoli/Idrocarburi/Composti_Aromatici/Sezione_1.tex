\section{Struttura del benzene}
Il \textbf{benzene} (\ch{C6H6}) è una molecola ciclica planare dove tutti gli atomi dell'anello sono ibridati \(sp^2\) e il restante orbitale \(2p\) puro è impiegato in un sistema di doppi legami coniugati. Il benzene, avendo tre doppi legami, ha proprietà molto diverse da quelle degli alcheni, per questo motivo è stato inserito in una classe di composti a se stante, i \textbf{composti aromatici}.

Il benzene e gli altri composti aromatici hanno le seguenti caratteristiche:
\begin{enumerate}
	\item Sono più stabili rispetto alle corrispondenti molecole a catena aperta
	\item La lunghezza dei legami dell'anello (1.40 \angstrom) è intermedia tra quella di un singolo (1.53 \angstrom) e di un doppio legame (1.33 \angstrom)
	\item Danno reazioni di sostituzione elettrofila piuttosto che di addizione elettrofila al doppio legame
\end{enumerate}


Il benzene viene rappresentato tramite le \textbf{strutture di Kekulé}, dove i tre doppi legami si spostano avanti e indietro così velocemente che le due forme di risonanza non possono essere separate, come rappresentato nella~\autoref{fig:risBenzene}. Ma può essere rappresentato anche con un cerchio al centro dell'esagono che indica lo spostamento dei doppi legami, come rappresentato nella~\autoref{fig:strBenzDel}.

\begin{figure}[H]
	\centering
	\begin{subfigure}[][][c]{.4\textwidth}
		\begin{center}
			\schemestart
			\chemleft[\subscheme{
					\chemfig{*6(-=-=-=)} \arrow{<->} \chemfig{*6(=-=-=-)}
				}\chemright]
			\schemestop
		\end{center}
		\caption{Strutture di Kekulé del benzene}\label{fig:risBenzene}
	\end{subfigure}
	\begin{subfigure}[][][c]{.4\textwidth}
		\begin{center}
			\schemestart
			\chemfig{**6(------)}
			\schemestop
		\end{center}
		\caption{Struttura del benzene }\label{fig:strBenzDel}
	\end{subfigure}
	\caption{Strutture del benzene}\label{fig:strBenzene}
\end{figure}





% \begin{minipage}{.5\textwidth}
% 	\begin{figure}[H]
% 		\centering
% 		\schemestart
% 		\chemleft[\subscheme{
% 				\chemfig{*6(-=-=-=)} \arrow{<->} \chemfig{*6(=-=-=-)}
% 			}\chemright]
% 		\schemestop
% 		\caption{Strutture di Kekulé del benzene}\label{fig:risBenzene}
% 	\end{figure}
% \end{minipage}
% \begin{minipage}{.5\textwidth}
% 	\begin{figure}[H]
% 		\centering
% 		\schemestart
% 		\chemfig{**6(------)}
% 		\schemestop
% 		\caption{Struttura del benzene }
% 	\end{figure}
% \end{minipage}
