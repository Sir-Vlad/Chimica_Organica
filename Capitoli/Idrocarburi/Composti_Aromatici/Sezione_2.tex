\newpage
\section{Aromaticità}
Abbiamo definito prima che il benzene è un composto aromatico, ma cosa si intende per aromatico?
Per definire se una molecola organica è aromatica bisogna che soddisfi queste condizioni:
\begin{enumerate}
	\item deve essere ciclica
	\item deve essere planare
	\item deve essere completamente coniugata
	\item deve soddisfare la \textbf{regola di H\"uckel}, ovvero deve contenere \((4n+2)\) elettroni \(\pi\)
\end{enumerate}

\paragraph{Molecole aromatiche non neutre}\mbox{}\\
Le molecole aromatiche non sono necessariamente neutre ma possono essere degli ioni, come per esempio, l'anione ciclopentadienile (\autoref{fig:AnCyclePentadinile}) e il catione cicloeptatrienile (\autoref{fig:CaCycleEptatrienile}).
\begin{figure}[H]
	\centering
	\begin{subfigure}{0.4\textwidth}
		\begin{center}
			\schemestart
			\chemfig{[:18]*5(-=-\charge{270:3pt=\:,45:3pt=\chargeColor{-}}{C}(-H)-=)}
			\schemestop
		\end{center}
		\caption{Anione ciclopentadienile}\label{fig:AnCyclePentadinile}
	\end{subfigure}
	\begin{subfigure}{0.4\textwidth}
		\begin{center}
			\schemestart
			\chemfig{[:-13]*7(-=-=-\charge{45:3pt=\chargeColor{+}}{C}(-H)-=)}
			\schemestop
		\end{center}
		\caption{Catione cicloeptatrienile}\label{fig:CaCycleEptatrienile}
	\end{subfigure}
	\caption{Anelli aromatici non neutri}
\end{figure}

L'anione ciclopentadienile è aromatico perché nei cinque orbitali \(p\) sono presenti 6 elettroni quindi rispetta la regola di H\"uckel. Mentre il catione cicloeptatrienile è aromatico perché nei sette orbitali \(p\) ci sono 6 elettroni.

\paragraph{Molecole aromatiche eterocicliche}\mbox{}\\
Possiamo applicare i requisiti di aromaticità anche ai \textbf{composti eterociclici} come la piridina o il pirrolo, altri composti eterociclici sono elencati nell'\autoref{ap:CompAromatici}. 
\begin{figure}[H]
	\centering
	\begin{subfigure}{0.4\textwidth}
		\begin{center}
			\schemestart
			\chemfig{*6(-=-=\charge{90:3pt=\:}{N}-=)}
			\schemestop
		\end{center}
		\caption{Piridina}
	\end{subfigure}
	\begin{subfigure}{0.4\textwidth}
		\begin{center}
			\schemestart
			\chemfig{[:18]*5(-=-\charge{270:3pt=\:}{N}(-H)-=)}
			\schemestop
		\end{center}
		\caption{Pirrolo}
	\end{subfigure}
	\caption{Anelli aromatici neutri}
\end{figure}
La piridina è molto simili al benzene differisce solo per una atomo di carbonio sostituito da uno di azoto, questo non influisce alla aromaticità perché l'azoto ha solo un elettrone nell'orbitale \(p\) come il carbonio. Mentre, il pirrolo è un anello a 5 termini con un atomo di azoto al suo interno, in questo caso il doppietto non condiviso dell'azoto viene usato nel sestetto aromatico.