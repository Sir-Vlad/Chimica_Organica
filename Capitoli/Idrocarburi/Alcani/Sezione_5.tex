\section{Isomeria \cis\texorpdfstring{\,\!}{} - \trans\texorpdfstring{\,\!}{} nei cicloalcani}
La \textbf{stereoisomeria} interessa tutte quelle molecole in cui gli atomi sono legati nella stessa sequenza ma hanno diversa disposizione nello spazio. L'\textbf{isomeria \cis-\trans\;\!} (detta anche \textbf{isomeria geometrica}) è un tipo di stereoisomeria.

\begingroup
\setchemfig{
	atom sep=3em,
	compound sep = 10em,
}
\begin{reaction*}
	\chemname{\chemfig[cram width=2pt]{?<[7,0.7](-[2,0.5]CH_3)-[,,,,line width=2pt](-[6,0.5]CH_3)>[1,0.7]-[:160]?}}{\iupac{\trans-1,2-dimetilciclopentano}}
	\arrow(.30--.150){-/>[non si][interconvertono]}
	\chemname{\chemfig[cram width=2pt]{?<[7,0.7](-[6,0.5]CH_3)-[,,,,line width=2pt](-[6,0.5]CH_3)>[1,0.7]-[:160]?}}{\iupac{\cis-1,2-dimetilciclopentano}}
\end{reaction*}
\endgroup

L'isomero \cis\;\! è quello con i sostituenti tutti dallo stesso lato mentre l'isomero \trans\;\! è quello che ha i sostituenti ai lati opposti. Questa differenza tra l'isomero \cis\;\! e l'isomero \trans\;\! porta delle sostanziali differenze in ambito di proprietà.  Bisogna sottolineare che i due isomeri non si interconvertono l'uno con l'altro come gli isomeri costituzionali, quindi possiamo dire che sono due molecole distinte. Questo si nota sopratutto, in ambito biologico, quando queste molecole reagiscono con gli enzimi.