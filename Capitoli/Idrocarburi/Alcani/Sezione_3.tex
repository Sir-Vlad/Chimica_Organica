\section{Cicloalcani}\label{sec:cicloalcani}
Un idrocarburo che contiene atomi di carbonio uniti in modo da formare un anello è chiamato \textit{idrocarburo ciclico}. Se tutti i legami sono tutti saturi, l'idrocarburo è chiamato  \textbf{cicloalcano}. In natura, sono presenti anelli da 3 a 30 atomi di carbonio, ma quelli più abbondanti sono quelli a 5 e 6 termini.

Gli anelli sono rappresentati generalmente come poligoni regolari aventi lo stesso numero di lati quanti sono gli atomi di carbonio.

\begin{table}[H]
	\centering
	\begin{tabular}{cccc}
		\chemfig{[:-30]*3(---)} & \chemfig{*4(----)} & \chemfig{[:17]*5(-----)} & \chemfig{[:30]*6(------)} \\
		Ciclopropano            & Ciclobutano        & Ciclopentano             & Cicloesano                \\
	\end{tabular}
\end{table}

La formula generale dei cicloalcani è \ch{C_{n}H_{2n}}. Il loro nome è il nome dell'idrocarburo a catena aperta con il prefisso \textit{ciclo-}.