\section{Struttura degli alcani}
% Tabella 3.1 libro
Il metano è il capostipite della famiglia degli alcani. Il carbonio del metano è ibridato \(sp^3\) e quindi la sua forma è tetraedrica e ha angoli di legame di \ang{109.5}. Di conseguenza, tutti gli altri alcani hanno la struttura tetraedrica e angoli approssimatamene di \ang{109.5}. I successivi alcani sono elencati nella~\autoref{tab:alcani}.

Per scrivere la formula degli alcani si può utilizzare diverse notazioni tra cui
\begin{itemize}
	\item la formula a linee e angoli \quad\(\longrightarrow\)\quad\chemfig{-[:+30,.6]-[:-30,.6]-[:+30,.6]-[:-30,.6]}
	\item la formula di struttura concisa
	      \begin{itemize}
		      \item \ch{CH2} raccolti \quad\(\longrightarrow\)\quad \ch{CH3(CH2)2CH3}
		      \item \ch{CH2} non raccolti \quad\(\longrightarrow\)\quad \ch{CH3CH2CH2CH3}
	      \end{itemize}
\end{itemize}

\begingroup
\begin{table}[H]
	\centering
	\rowcolors{2}{gray!15}{}
	\renewcommand{\arraystretch}{1.2}
	\begin{tabular}{ll|ll}
		\specialrule{.1em}{.05em}{.05em}
		\textbf{Nome} & \textbf{Formula di struttura} & \textbf{Nome} & \textbf{Formula di struttura} \\
		\specialrule{.1em}{.05em}{.05em}
		metano        & \ch{CH4}                      & undecano      & \ch{CH3(CH2)9CH3}             \\
		etano         & \ch{CH3CH3}                   & dodecano      & \ch{CH3(CH2)10CH3}            \\
		propano       & \ch{CH3CH2CH3}                & tridecano     & \ch{CH3(CH2)11CH3}            \\
		butano        & \ch{CH3(CH2)2CH3}             & tetradecano   & \ch{CH3(CH2)12CH3}            \\
		pentano       & \ch{CH3(CH2)3CH3}             & pentadecano   & \ch{CH3(CH2)13CH3}            \\
		esano         & \ch{CH3(CH2)4CH3}             & esadecano     & \ch{CH3(CH2)14CH3}            \\
		eptano        & \ch{CH3(CH2)5CH3}             & eptadecano    & \ch{CH3(CH2)15CH3}            \\
		ottano        & \ch{CH3(CH2)6CH3}             & ottadecano    & \ch{CH3(CH2)16CH3}            \\
		nonano        & \ch{CH3(CH2)7CH3}             & nonadecano    & \ch{CH3(CH2)17CH3}            \\
		decano        & \ch{CH3(CH2)8CH3}             & eicosano      & \ch{CH3(CH2)18CH3}            \\
		\specialrule{.1em}{.05em}{.05em}
	\end{tabular}
	\caption{Nomi e struttura abbreviata dei primi 20 alcani lineari}\label{tab:alcani}
\end{table}
\endgroup

Gli alcani hanno formula generale \ch{C_{n}H_{2n+2}}. Così, dato il numero di atomi di carbonio ci possiamo subito ricavare il numero di idrogeni nella molecola e anche la formula molecolare.