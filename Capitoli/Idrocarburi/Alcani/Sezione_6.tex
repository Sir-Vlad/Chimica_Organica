\section{Proprietà fisiche di alcani e cicloalcani}
La proprietà più importante degli alcali è di essere non polari, questo deriva dalla differenza di elettronegatività tra \elementsymbol{1} e \elementsymbol{6} che è di 0.4.

\subsection{Punto di ebollizione}
I punti di ebollizione aumentano all'aumentare del peso molecolare, ma sono molto più bassi rispetto a altre molecole dello stesso peso molecolare.

Gli alcani da 1 a 4 atomi di carbonio sono gas a temperatura ambiente, quelli da 5 a 17 atomi di carbonio sono liquidi mentre quelli da 18 atomi in su sono solidi cerosi di colore bianco.

\begin{table}[H]
	\centering
	\rowcolors{2}{gray!15}{}
	\renewcommand{\arraystretch}{1.2}
	\begin{tabular}{llccc}
		\toprule
		\rowcolors{1}{white}{} & \multicolumn{1}{c}{\textbf{Formula di}} & \textbf{Punto di} & \textbf{Punto di}    &                                                           \\
		                       & \multicolumn{1}{c}{\textbf{struttura}}  & \textbf{fusione}  & \textbf{ebollizione} & \textbf{Densità del liquido}                              \\
		\textbf{Nome}          & \multicolumn{1}{c}{\textbf{abbreviata}} & (\unit{\celsius}) & (\unit{\celsius})    & (\unit[per-mode = symbol]{\g\per\ml} a \qty{0}{\celsius}) \\

		\midrule
		metano                 & \ch{CH4}                                & -182              & -164                 & (gas)                                                     \\
		etano                  & \ch{CH3CH3}                             & -183              & -88                  & (gas)                                                     \\
		propano                & \ch{CH3CH2CH3}                          & -190              & -42                  & (gas)                                                     \\
		butano                 & \ch{CH3(CH2)2CH3}                       & -138              & 0                    & (gas)                                                     \\
		pentano                & \ch{CH3(CH2)3CH3}                       & -130              & 36                   & 0.626                                                     \\
		esano                  & \ch{CH3(CH2)4CH3}                       & -95               & 69                   & 0.659                                                     \\
		eptano                 & \ch{CH3(CH2)5CH3}                       & -90               & 98                   & 0.684                                                     \\
		ottano                 & \ch{CH3(CH2)6CH3}                       & -57               & 126                  & 0.703                                                     \\
		nonano                 & \ch{CH3(CH2)7CH3}                       & -51               & 151                  & 0.718                                                     \\
		decano                 & \ch{CH3(CH2)8CH3}                       & -30               & 174                  & 0.730                                                     \\
		\bottomrule
	\end{tabular}
	\caption{Proprietà di alcuni alcani lineari}\label{tab:proprieta_alcani}
\end{table}

\subsection{Forze di dispersione e interazioni tra alcani}
Nelle molecole degli alcani non c'è separazione di carica, però in ogni instante si può creare uno squilibrio nella nube elettronica, creando dei dipoli indotti. Questi dipoli indotti creano tra le molecole degli alcani delle forze di dispersione, che sono forze attrattive di debole intensità tra cariche istantanee su atomi o molecole vicine.

Queste forze aumentano all'aumentare del peso molecolare, e di conseguenza anche il punto di ebollizione aumenta.

\subsection{Punti di fusione e densità}
Anche i punti di fusione aumentano all'aumentare del peso molecolare, però questo non del tutto regolare come i punti di ebollizione perché la capacità di impacchettarsi su schemi ordinati cambia a seconda della dimensione molecolare e dalla forma.

La densità di alcuni alcani è elencata nell'\autoref{tab:proprieta_alcani}, in media è di \qty{0.7}{\g\per\ml} perciò tutti gli alcani galleggiano in acqua.

\subsection{Isomeri costituzionali e le loro proprietà}
Gli isomeri costituzionali di un alcano sono molecole differenti con proprietà proprie. Per ogni gruppo di isomeri costituzionali si è notato che l'isomero meno ramificato possiede il punto di ebollizione più basso rispetto a quelli più ramificati. Questo effetto succede perché l'isomero meno ramificato avrà più interazioni intermolecolari rispetto ai suoi isomeri più ramificati.

\begin{table}[H]
	\centering
	\rowcolors{2}{gray!15}{}
	\renewcommand{\arraystretch}{1.2}
	\begin{tabular}{lccc}
		\toprule
		\rowcolors{1}{white}{} & \textbf{Punto di} & \textbf{Punto di}    &                                                           \\
		                       & \textbf{fusione}  & \textbf{ebollizione} & \textbf{Densità del liquido}                              \\
		\textbf{Nome}          & (\unit{\celsius}) & (\unit{\celsius})    & (\unit[per-mode = symbol]{\g\per\ml} a \qty{0}{\celsius}) \\
		\midrule
		\iupac{esano}                  & -95               & 69                   & 0.659                                                     \\
		\iupac{3-metilpentano} & -6 & 64& 0.664\\
		\iupac{2-metilpentano} & -23 & 62 & 0.653\\
		\iupac{2,3-dimetilbutano} & -129 & 58 & 0.662\\
		\iupac{2,2-dimetilbutano} & -100 & 50 & 0.649\\
		\bottomrule
	\end{tabular}
\end{table}