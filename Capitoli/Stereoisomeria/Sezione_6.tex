\section{Miscele racemiche}
Una miscela equimolecolare di due enantiomeri è chiamata \textbf{miscela racemica}. Poiché la miscela racemica contiene un uguale numero di molecole destrogire e levogire, la sua rotazione è \ang{0}.

\subsection{Risoluzione delle miscele racemiche}
La separazione di una miscela racemica nei suoi enantiomeri è detta \textbf{risoluzione}. La separazione di enantiomeri è, in generale, difficile. Per risolvere questo problema, i due enantiomeri vengono trasformati in diastereoisomeri, separati e poi ritrasformati in enantiomeri.

Per separare due enantiomeri dobbiamo farmi reagire con un reagente anch'esso chirale. Il prodotto sarà una coppia di diastereoisomeri che si possono separare molto facilmente.
\begin{equation*}
	\underset{\substack{\text{coppia di}\\\text{enantiomeri}}}{
		\begin{Bmatrix}R\\S\end{Bmatrix}}
	\quad + \quad \underset{\substack{\text{reagente}\\\text{chirale}}}{R}
	\quad\longrightarrow\quad
	\underset{\substack{\text{miscela di}\\\text{diastereoisomeri}}}{
		\begin{Bmatrix}\ch{R-R}\\\ch{S-R}\end{Bmatrix}}
\end{equation*}
Dopo aver separato i due diastereoisomeri, dobbiamo farli reagire in modo da recuperare il reagente chirale e ottenere l'enantiomero.
\begin{gather*}
	\ch{R-R} \quad\longrightarrow\quad R \quad + \quad R\\
	\ch{S-R} \quad\longrightarrow\quad S \quad + \quad R
\end{gather*}

In molte reazioni biologiche si riescono a separare gli enantiomeri tramite un enzima (\textit{reagente chirale}) che è affine solo a uno dei due enantiomeri, facendo così si riesce a separarli. 