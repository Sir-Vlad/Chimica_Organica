\chapter{Stereochimica}
% Introduzione
La \textbf{stereochimica} studia le proprietà spaziali delle molecole (assenza o presenza di centri, piani e assi di simmetria riflessiva o rotazionale) e come queste ultime si riflettano sul comportamento chimico delle sostanze.

In particolare, la stereochimica organica studia la simmetria delle molecole organiche, la loro chiralità, la relazione tra chiralità e stereogenicità, l'interazione tra molecole chirali, la sintesi di sostanze otticamente pure e la separazione di stereoisomeri.

\subimport*{./}{Sezione_1.tex} % Stereoisomeria
\subimport*{./}{Sezione_2.tex} % Denominazione stereocentri: sistema R e S
\subimport*{./}{Sezione_3.tex} % Diastereoisomeria
\subimport*{./}{Sezione_5.tex} % Attività ottica
\subimport*{./}{Sezione_6.tex} % Separazione enantiomeri