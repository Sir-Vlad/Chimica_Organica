\vspace{-20pt}
\section{Diastereoisomeria}
I composti possono avere più di un centro stereocentro, in tal caso il numero di stereoisomeri sarà \(2^n\) dove \(n\) rappresenta il numero di stereocentri che la molecola possiede.

Prendiamo com esempio, una molecola con due stereocentri. Ogni stereocentro può avere la configurazione \textit{R} e \textit{S}, perciò sono possibili in tutto quattro isomeri: \cip{R,R}, \cip{S,S},\cip{R,S} e \cip{S,R}. Osserviamo che ci sono due coppie di enantiomeri, infatti le forme \cip{R,R} e \cip{S,S} sono immagini speculari non sovrapponibili ma anche le forme \cip{R,S} e \cip{S,R} lo sono.
Tra le due coppie notiamo che solo per un centro stereogeno hanno la stessa configurazione mentre l'altro è diverso quindi non possono essere enantiomeri. Questo tipo di stereoisomeri prendono il nome di \textbf{diasteroisomeri}.

\subsection{Composti \texorpdfstring{\meso}{meso}}
I \textbf{composti \meso} sono molecole con più stereocentri dove i centri stereogeni sono legati con gli stessi gruppi e questo gli fa perdere la chiralità perché si può individuare un'asse di simmetria.

Un esempio di questo fenomeno è la molecola di~\iupac{acido tartarico}:
{\small\begin{center}
	\chemname{\chemfig{\chemabove{C}{\color{red}*}H(-[4]HOC(=[2,,3]O))(-[6]OH)-\chemabove{C}{\color{red}*}H(-[0]COH(=[2]O))(-[6]OH)}}{\iupac{Acido 2,3-diidrossibutandioico}\\(\iupac{Acido tartarico})}
\end{center}}
La quale avendo due centri stereogeni possiamo scrivere quattro stereoisomeri:
\begin{center}
	\setchemfig{
		atom sep=1.75em
	}
	\begin{tikzpicture}[node distance=1.5cm,every node/.style={font=\sffamily}, align=center]
		\node (RR) {\chemfig{(-[2]COOH)(-[0]OH)(-[4]H)-[6](-[6]COOH)(-[4]HO)(-[0]H)}};
		\node (labelRR) [below of=RR,yshift=-8pt] {\cip{R,R}};
		\node (SS) [right of=RR,xshift=1.5cm] {\chemfig{(-[2]COOH)(-[4]HO)(-[0]H)-[6](-[6]COOH)(-[0]OH)(-[4]H)}};
		\node (labelSS) [below of=SS,yshift=-8pt] {\cip{S,S}};
		\draw ($(RR.north) !0.5! (SS.north)$) -- ($(RR.south) !0.5! (SS.south)$);
		\draw[decorate,decoration={brace,mirror,amplitude=0.3cm,raise=0.6cm}] (RR.south west) -- node[below,yshift=-.8cm] {enantiomeri, chirali} (SS.south east);

		\node (RS) [right of=SS,xshift=3cm] {\chemfig{(-[2]COOH)(-[0]OH)(-[4]H)-[6](-[6]COOH)(-[0]OH)(-[4]H)}};
		\node (labelRS) [below of=RS,yshift=-8pt] {\cip{R,S}};
		\node (SR) [right of=RS,xshift=1.5cm] {\chemfig{(-[2]COOH)(-[4]HO)(-[0]H)-[6](-[6]COOH)(-[4]HO)(-[0]H)}};
		\node (labelSR) [below of=SR,yshift=-8pt] {\cip{S,R}};
		\draw ($(RS.north) !0.5! (SR.north)$) -- ($(RS.south) !0.5! (SR.south)$);
		\draw[dashed,blue] (RS.west) -- (RS.east);
		\draw[dashed,blue] (SR.west) -- node[above,xshift=2cm] {piano di} node[below,xshift=2cm] {simmetria} +(5.5cm,0);
		\draw[decorate,decoration={brace,mirror,amplitude=0.3cm,raise=0.6cm}] (RS.south west) -- node[below,yshift=-.8cm] {strutture identiche,\\achirali, forma \meso} (SR.south east);
	\end{tikzpicture}
\end{center}
In questo caso gli isomeri \cip{R,R} e \cip{S,S} costituiscono una coppia di enantiomeri mentre gli isomeri \cip{S,R} e \cip{R,S} sono la stessa molecola solo ruotata di \ang{180}.