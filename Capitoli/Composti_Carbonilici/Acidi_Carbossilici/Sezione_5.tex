\section{Esteri}\label{sec:esteri}

Gli \textbf{esteri} derivano per sostituzione del gruppo \ch{-OH} con il gruppo \ch{-OR} o \ch{-OAr}. Esistono anche gli esteri ciclici e vengono chiamati \textbf{lattoni}.
\begin{figure}[H]
	\centering
	\setlength{\tabcolsep}{1cm}
	\renewcommand{\arraystretch}{2}
	\begin{tabular}{ccc}
		\chemfig{H_3C-[:30]C(=[2]O)-[:-30]O-[:30]CH_3}
		                  &
		\chemfig{C(-[:210]*6(-=-=-=))(=[2]O)-[:-30]O-[:30]CH_3}
		                  & \chemfig{[:288]*5(--O-(=O)--)}                            \\
		Acetato di metile & Benzoato di metile             & \iupac{\g-Butanolattone} \\
	\end{tabular}
\end{figure}

\subsection{Idrolisi degli estri}
L'idrolisi degli esteri è molto lenta a \pH neutro anche se riscaldata. Per questo motivo viene eseguita a \pH acido o basico e riscaldata.

	% \newcommand{\circleNum}[1]{} comando numeri sulle freccie
	{\small\begin{reaction}
			\AddRxnDesc{Idrolisi acida degli esteri}
			\arrow{0}[,0]
			% - 1 reazione
			\chemfig{!\DerAc O-[:30]R}
			\arrow{<=>[\Hpiu[m]{1}][\circleNum{1}]}
			% - 2 reazione
			\chemfig{R-[:30]@{C2}C(=[@{Ol2}2]@{O2}\charge{135:3pt=\chargeColor{+}}{O}H)-[:-30]O-[:30]R}
			\+
			\chemfig{H-[:30]@{O3}\charge{90=\:,270=\:}{O}-[:-30]H}
			\arrow{<=>[][\circleNum{2}]}
			% - 3 reazione
			\chemfig{R-C(-[2]OH)(-[6]@{O4}\charge{135:3pt=\chargeColor{+}}{O}(-[@{Hl2}:210]H)-[:-30]H)-[0]O-[:30]R}
			\arrow{<=>[*{0}\chemfig{@{H2}\charge{45:3pt=\chargeColor{+}}{H}}][*{0}\circleNum{3}]}[-90]
			% - 4 reazione
			\chemfig{R-C(-[2]OH)(-[6]O-[:-30]H)-[0]@{O5}\charge{90=\:,270=\:}{O}-[:30]R}
			\arrow{<=>[][\circleNum{4}]}[-180]
			% - 5 reazione
			\chemfig{R-C(-[@{Ol6}2]@{O6}\charge{90=\:,180=\:}{O}H)(-[6]O-[:-30]H)-[@{Ol7}0]@{O7}\charge{90:3pt=\chargeColor{+}}{O}(-[:-30]H)-[:30]R}
			\arrow{<=>[][\circleNum{5}]}[-180]
			% - 6 reazione
			\chemfig{R-[:30]C(=[2]@{O8}\charge{90=\:,180=\:,135:3pt=\chargeColor{+}}{O}-[@{Hl3}:30]H)-[:-30]O-[:30]H}
			\arrow{<=>[][\circleNum{6}]}[-180]
			\chemfig{R-[:30]C(=[2]O)-[:-30]O-[:30]H}
			\chemmove[green!60!black!70]{
				\draw[shorten <=3pt, shorten >= 1pt] (O).. controls +(45:1cm) and +(135:1cm) .. (H1);
				\draw[shorten <=3pt, shorten >= 1pt] (O3).. controls +(270:1.3cm) and +(270:1.3cm) .. (C2);
				\draw[shorten <=3pt, shorten >= 1pt] (Ol2).. controls +(180:.5cm) and +(180:.5cm) .. (O2);
				\draw[shorten <=3pt, shorten >= 1pt] (Hl2).. controls +(-50:.5cm) and +(270:.5cm) .. (O4);
				\draw[shorten <=3pt, shorten >= 1pt] (O5).. controls +(70:1cm) and +(-20:1cm) .. (H2);
				\draw[shorten <=3pt, shorten >= 1pt] (O6).. controls +(180:.5cm) and +(180:.5cm) .. (Ol6);
				\draw[shorten <=3pt, shorten >= 1pt] (Ol7).. controls +(270:.5cm) and +(270:.5cm) .. (O7);
				\draw[shorten <=3pt, shorten >= 1pt] (Hl3).. controls +(-45:.5cm) and +(-15:.5cm) .. (O8);
			}
		\end{reaction}}

\begin{description}
	\item[Stadio 1:] Protonazione dell'ossigeno carbonilico con aumento dell'elettrofilicità del carbonio carbonilico
	\item[Stadio 2:] Attacco dell'acqua al carbonio carbonilico con successivo ribaltamento del doppio legame sull'ossigeno carbonilico e formazione dell'intermedio tetraedrico
	\item[Stadio 3:] Deprotonazione dell'ossigeno
	\item[Stadio 4:] Protonazione dell'ossigeno estereo
	\item[Stadio 5:] Riformazione del doppio legame carbonilico carbonio-ossigeno e uscita del gruppo alcolico
	\item[Stadio 6:] Deprotonazione dell'ossigeno carbonilico
\end{description}

\subsection{Saponificazione degli esteri}
L'idrolisi può essere effettuata anche usando una base acquosa a caldo come \ch{NaOH}, questa reazione viene spesso chiamata \textbf{saponificazione}. Il meccanismo di reazione è molto simile a quello dell'idrolisi in ambiente acido. L'unica differenza sta nello stadio 3 dove la reazione diventa irreversibile.
	{\small\begin{reaction}
			\AddRxnDesc{Saponificazione degli esteri}
			\arrow{0}[,0]
			% - 1 reazione
			\chemfig{R-[:30]@{C1}C(=[@{Ol1}2]@{O1}O)-[:-30]O-[:30]R}
			\arrow{0}[,0]\+
			\chemfig{@{O2}\charge{90=\:,180=\:,270=\:,135:3pt=\chargeColor{-}}{O}-H}
			\arrow{<=>[][\circleNum{1}]}
			% - 2 reazione
			\chemfig{R-C(-[@{Ol3}2]@{O3}\charge{0=\:,90=\:,180=\:,45:3pt=\chargeColor{-}}{O})(-[6]OH)-[@{Ol4}0]@{O4}O-[:30]R}
			\arrow{<=>[][\circleNum{2}]}
			% - 3 reazione
			\subscheme{
				\chemfig{!\DerAc OH} \arrow{0}[,0]\+ \chemfig{\charge{90=\:,180=\:,270=\:,135:3pt=\chargeColor{-}}{O}-R}
			}
			\arrow{->[*{0}\circleNum{3}]}[-90]
			\subscheme{
				\chemfig{!\DerAc \ChargeOop} \arrow{0}[,0]\+ \chemfig{R-OH}
			}
			\chemmove[green!60!black!70]{
				\draw[shorten <=3pt, shorten >= 1pt] (O2).. controls +(270:1.3cm) and +(270:1.3cm) .. (C1);
				\draw[shorten <=3pt, shorten >= 1pt] (Ol1).. controls +(180:.5cm) and +(180:.5cm) .. (O1);
				\draw[shorten <=3pt, shorten >= 1pt] (O3).. controls +(0:.5cm) and +(0:.5cm) .. (Ol3);
				\draw[shorten <=3pt, shorten >= 1pt] (Ol4).. controls +(-90:.5cm) and +(-90:.5cm) .. (O4);
			}
		\end{reaction}}

\subsection{Amminolisi degli esteri}
L'\textbf{amminolisi} degli esteri trasforma l'estere in un'ammide. La reazione può avvenire con ammoniaca o con ammine primarie e secondarie ma non terziarie.
\begin{reaction}
	\arrow{0}[,0]
	\AddRxnDesc{Amminolisi degli esteri}
	\chemfig{!\DerAc O-[:30]R} \arrow{0}[,0]\+ \chemfig{R-NH_2}
	\arrow
	\chemfig{!\DerAc NHR} \arrow{0}[,0]\+ \chemfig{R-OH}
\end{reaction}

Il meccanismo è molto simile a quello della saponificazione. Il doppietto non condiviso dell'azoto è responsabile dell'attacco iniziale al carbonio estereo.

\subsection{Transesterificazione degli esteri}
La \textbf{transesterificazione} è la trasformazione di un estere in un altro estere per reazione con un alcol. Il meccanismo è uguale all'idrolisi degli esteri in ambiente acido.

\begin{reaction}
\AddRxnDesc{Transesterificazione degli esteri}
\arrow{0}[,0]
\chemfig{!\DerAc O-[:30]R'} \arrow{0}[,0]\+ \chemfig{R''-OH}
\arrow
\chemfig{!\DerAc O-[:30]R''} \arrow{0}[,0]\+ \chemfig{R'-OH}
\end{reaction}


\subsection{Riduzione degli esteri}
Gli esteri possono essere ridotti ad alcoli primari tramite idruro di alluminio e litio (\ch{LiAlH4}) e da come prodotti un alcol primario e un secondo alcol.
\chemnameinit{}
\begin{reaction}
	\AddRxnDesc{Riduzione degli esteri}
	\arrow{0}[,0]
	\chemfig{!\DerAc O-[:30]R} \arrow(--.mid west){->[\ch{LiAlH4}][etere]}[,1.3] \chemname{\chemfig{R-[:30]CH_2-[:-30]OH}}{Alcol primario} \+ \chemfig{R-OH}
\end{reaction}
\chemnameinit{}