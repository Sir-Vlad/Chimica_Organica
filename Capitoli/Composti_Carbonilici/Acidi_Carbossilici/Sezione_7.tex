\section{Ammidi}\label{sec:ammidi}
% Ammidi
% idrolidi delle ammidi

Le \textbf{ammidi} derivano per sostituzione del gruppo \ch{-OH} con il gruppo \ch{-NR_2}, dove \ch{R} può essere un gruppo alifatico o un idrogeno. Le ammidi cicliche vengono chiamate \textbf{lattami}.
\begin{figure}[H]
	\centering
	\setlength{\tabcolsep}{.5cm}
	\renewcommand{\arraystretch}{1.3}
	\begin{tabular}{cc}
		\chemfig{H_3C-[:30]C(=[2]O)-[:-30]NH_2}               & \chemfig{H_3C-[:30]C(=[2]O)-[:-30]N(-[1]H)(-[7]CH_3)} \\
		\iupac{Acetammide}                                    & \iupac{\N-metilacetammide}                            \\
		\textit{Ammide primaria}                              & \textit{Ammide secondaria}                            \\
		\chemfig{H-[:30]C(=[2]O)-[:-30]N(-[1]CH_3)(-[7]CH_3)} & \chemfig{*4((<CH_3)-NH-(=O)--)}                       \\
		\iupac{\N,\N-Dimetilformammide (DMF)}                 & \iupac{\S-\b-butanolattame}                           \\
		\textit{Ammide terziaria}                             & \textit{Ammide ciclica - Lattame}                     \\
	\end{tabular}
\end{figure}

\subsection{Idrolisi delle ammidi}
Le ammidi per essere idrolizzate necessitano di condizioni di reazioni molto spinte rispetto agli esteri.
\begingroup
	\setchemfig{arrow label sep=5pt}
	\begin{reaction}
		\AddRxnDesc{Idrolisi delle ammidi}
		\arrow{0}[,0]
		\chemfig{!\DerAc NH_2} \arrow{0}[,0]\+ \chemfig{H_2O} 
		\arrow{->[\Hpiu{1} o][\chemfig{H\charge{25:3pt=\chargeColor{-}}{O}}]}[,1.5]
		\chemfig{!\DerAc OH} \arrow{0}[,0]\+ \chemfig{NH_3}
	\end{reaction}
\endgroup

% ADD: riduzione delle ammidi