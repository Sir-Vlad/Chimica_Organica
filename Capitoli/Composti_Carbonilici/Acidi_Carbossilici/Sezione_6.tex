\section{Anidridi}\label{sec:anidridi}
Le \textbf{anidridi} derivano dall'unione di due acidi carbossilici per perdita di una molecola di acqua. Le anidridi possono essere: \textbf{simmetriche}, due gruppi acilici uguali, o \textbf{miste}, due gruppi aciclici diversi.
\begin{figure}[H]
	\centering
	\setlength{\tabcolsep}{1cm}
	\renewcommand{\arraystretch}{2}
	\begin{tabular}{cc}
		\chemfig{H_3C-[:30]C(=[2]O)-[:-30]O-[:30]C(=[2]O)-[:-30]CH_3}
		&
		\chemfig{C(-[:210]*6(-=-=-=))(=[2]O)-[:-30]O-[:30]C(-[:-30]*6(-=-=-=))(=[2]O)}\\
		Anidride acetica & Anidride benzoica\\
	\end{tabular}	
\end{figure}


Le anidridi possono essere preparate dai \hyperref[sec:cloruriAcilici]{cloruri acilici} e dai sali degli acidi carbossilici tramite sostituzione nucleofila acilica. Questo è il metodo per preparare le \textbf{anidridi miste}.

\begin{reaction}
	\AddRxnDesc{Preparazione anidridi miste}
	\arrow{0}[,0]
	\chemfig{!\DerAc Cl} \+{,,.7cm} \chemfig{!\DerAc \ChargeOop-[,0.8,,,draw=none]\charge{45:3pt=\chargeColor{+}}{Na}}
	\arrow
	\chemfig{!\DerAc O-[:30]C(=[2]O)-[:-30]R}
\end{reaction}

\subsection{Reazioni delle anidridi}
Le reazioni caratteristiche delle anidridi sono le seguenti, e seguono tutte lo stesso meccanismo visto prima:

\tikzset{
arrow/.style={->,>={Latex[width=2mm,length=2mm]}}
}


{\footnotesize
\begin{reaction}
	\AddRxnDesc{Reazioni delle anidridi}
	\arrow{0}[,0]
	\chemfig{!\DerAc O-[:30]C(=[2]O)-[:-30]R}
	\arrow{0}[,2.5,red] % - freccie di costruzione non visibili
	\chemfig{!\DerAc OH} \+{,,0.8cm} \chemfig{!\DerAc OH}
	\arrow(@c3--){0}[90] % - freccie di costruzione non visibili
	\chemfig{!\DerAc OR} \+{,,0.8cm} \chemfig{!\DerAc OH}
	\arrow(@c3--){0}[-90] % - freccie di costruzione non visibili
	\subscheme{
		\chemfig{!\DerAc NH_2} \+{,,0.8cm} \chemfig{!\DerAc OH}
	}
	\chemmove[shorten <=3pt, shorten >= 3pt]{
		\draw[arrow] (c2) -- node[midway,yshift=10pt,xshift=12pt] {\chemfig{H_2O}} (c3);
		\draw[arrow] (c2) -- +(2.8,0) |- node[near end,yshift=10pt] {\chemfig{ROH}} (c4);
		\draw[arrow] (c2) -- +(2.8,0) |- node[near end,yshift=10pt] {\chemfig{RNH_2}} (c6);
	}
\end{reaction}}
L'acqua idrolizza le anidridi per fornire l'acido corrispondente. Gli alcoli reagendo formando esteri, mentre con l'ammoniaca o le ammine si ottengono ammidi. In tutte queste reazioni si ottiene anche un equivalente di acido.

