\section{Struttura}

Gli aldeidi e i chetoni sono composti caratterizzati dalla presenza del \textbf{gruppo carbonile}. Le aldeide hanno il carbonile in posizione terminale, mentre nei i chetoni si trova all'interno della catena. Il gruppo \ch{-CH=O} delle aldeidi si chiama \textbf{gruppo formilico}.

\begin{figure}[H]
	\centering
	\setlength{\tabcolsep}{1cm}
	\renewcommand{\arraystretch}{2}
	\tikzset{wv/.style={decorate,decoration=complete sines}}
	\begin{tabular}{cccc}
		\chemfig{C(=[2]O)(-[:210])(-[:-30])} & \chemfig{C(=[2]O)(-[:210]R)(-[:-30]H)} & \chemfig{C(=[2]O)(-[:210]R)(-[:-30]R)} & \chemfig{C(=[2]O)(-[:210,0.7]((-[:120,0.5,,,wv])-[:-60,0.5,,,wv]))(-[:-30]H)} \\
		carbonile                            & aldeide                                & chetone                                & formile                                                                       \\
	\end{tabular}
\end{figure}

Il carbonile ha struttura planare perché il carbonio è ibridato \(sp^2\). Il carbonio del carbonile ha una parziale carica positiva sia per effetto induttivo, dovuto all'elettronegatività dell'ossigeno, sia per risonanza.

\begin{center}
	\setlength{\tabcolsep}{1cm}

	\begin{tabular}{cc}
		\schemestart
		\chemleft[\subscheme{
		\chemfig{@{C}C(=[@{O}2]O)(-[:210])(-[:-30])} \arrow{<->}
		\chemfig{\charge{30:3pt=\chargeColor{+}}{C}(-[2]\charge{30:3pt=\chargeColor{-}}{O})(-[:210])(-[:-30])}
		}\chemright]
		\schemestop
		\chemmove[green!60!black!70]{
			\draw[shorten <=2pt, shorten >= 2pt] (C).. controls +(150:.5cm) and +(180:.5cm) .. (O);
		} &
		\chemfig{\charge{30:4pt=\color{red}\(\scriptstyle\delta^+\)}{C}(=[2]\charge{30:4pt=\color{blue}\(\scriptstyle\delta^-\)}{O})(-[:210])(-[:-30])} \\
	\end{tabular}
\end{center}

\noindent Le molecole di aldeidi e chetoni hanno tre punti reattivi: ossigeno, carbonio e \(\alpha\)-idrogeno.
\begin{enumerate}
	\item L'ossigeno del carbonio può protonarsi in ambiente acido
	\item Il carbonio del carbonile può subire addizione nucleofila
	\item L'idrogeno in posizione \(\alpha\) è parzialmente acido è può essere strappato nelle reazioni cheto-enoliche. (\autoref{sec:tautomeria})
\end{enumerate}

\begin{figure}[H]
	\centering
	\schemestart
	\chemfig{@{C}C(=[2]@{O}O)(-[:210]C(-[6]@{Ha}H)(-[:150]))(-[:-30])}
	\arrow(@c1--){0}[45,0.5] \chemfig{@{Hc}\charge{35:3pt=\chargeColor{+}}{H}}
	\arrow(@c1--){0}[20,0.5] \chemfig{@{nu}\charge{180:2pt=\:}{Nu}}
	\arrow(@c1--){0}[180,0.5] \chemfig{@{B}\charge{0:2pt=\:}{B}}
	\schemestop
	\chemmove[green!60!black!70]{
		\draw[shorten <=3pt, shorten >= 1pt] (O).. controls +(90:1cm) and +(180:1cm) .. (Hc);
		\draw[shorten <=5pt, shorten >= 1pt] (nu).. controls +(180:1cm) and +(0:1cm) .. (C);
		\draw[shorten <=5pt, shorten >= 1pt] (B).. controls +(0:1cm) and +(180:1cm) .. (Ha);
	}
\end{figure}

A seguito della polarizzazione del legame \ch{C=O}, la maggior parte delle reazioni dei composti carbonilici comporta l'attacco di un nucleofilo sull'atomo di carbonio carbonilico.
% In aggiunta, la polarizzazione del legame \ch{C=O} influenza le proprietà fisiche dei composti carbonilici come il punto di ebollizione o la solubilità.

% Il punto di ebollizione di aldeidi e chetoni è più alto degli idrocarburi ma più basso degli alcoli. Questo è dovuto dalla polarità del legame \ch{C=O} che polarizza la molecola. Questa polarizzazione tende ad associarsi la parte positiva di una molecola con la parte negativa di un'altra. I p.e. sono più bassi degli alcoli perché non avendo gruppi \ch{-OH} non possono creare legami idrogeno tra le molecole ma li possono accettare, questo fa si che composti carbonilici a basso peso molecolare sono solubili in acqua.
La polarizzazione del legame \ch{C=O} influenza anche il punto di ebollizione e la solubilità. Il punto di ebollizione di aldeidi e chetoni è più alto degli idrocarburi ma più basso degli alcoli. Questo avviene perché le aldeidi e i chetoni non avendo gruppi ossidrilici \ch{-OH} non possono creare legami a idrogeno ma possono solo accettarli per la presenza dell'ossigeno carbonilico. Visto che possono solo accettare legami a idrogeno, i composti carbonilici a basso peso molecolare sono solubili in \ch{H2O}.